\section{Abast}
L'objectiu del projecte es desenvolupar un conjunts de serveis webs per manegar l'algoritme de Backtracking bacteriol\`{o}gic Ichnaea. Actualment Ichnaea es troba en la versi\'{o} 2.0, desenvolupat per Aitor P\'{e}rez P\'{e}rez. La primera versi\'{o} va ser desenvolupada per David Sanchez.\\

La complexitat de les entrades i configuracions dels parametres de entrada de Ichnaea, requereixen de unes interf\'{i}cies i d'un model de dades per poder executar l'algoritme de forma m\'{e}s amigable i comprensible. El prop\`{o}sit del projecte es dissenyar e implementar aquest sistema en un entorn distribu\¨{i}t en xarxa.\\

En col.laboraci\'{o} amb Miguel Ibero, que desenvolupa com a Projecte de Final de Carrera un sistema de cues per manegar les execucions de Ichnaea, integrarem una primera versi\'{o}.\\


\section{Escenari, entitats i l'univers Ichnaea}
Per tal de poder entendre els objectius del projecte farem una visi\'{o} global de Ichnaea i els seus elements.\\

\subsection{Matrius}
Ichnaea processa inicialment les dades a partir de unes matrius on \'e{s} defineixen les dades per variables i per mostres. Aquest conjunt de valors, a partir de ara en tota la documentaci\'{o} seran anomenades "matrius", son els valors per a:\\
\begin{itemize}
\item Variables: son les columnes de la matriu. 
\item Mostres: son les files de la matriu. A partir d'ara en tota la documentaci\'{o} s'anomenaran "samples".
\end{itemize}

\subsubsection{Variable de la matriu i conjunts de envelliments: Season Set i Seasons}
Una variable de la matriu pot ser una:\\
\begin{itemize}
\item Una variable, per exemple, de una bacteria: "variable single"
\item Una variable derivada de un calcul entre dos "variables single"
\item Una variable de origen. Obligatoria per cada "sample".
\end{itemize}
Les "variables single" tenen associades unes regresions, representat en un fitxer, on s'especifica els envelliments. Un conjunt de regresions els anomenem "Season Sets". A un component d'aquest conjunt de regressions l'anomenarem "Season".\\

Les variables de origen representen una etiqueta de la mostra per identificar l'origen de la mostra.\\

\subsection{Trainings: els entrenaments}
Ichnaea utilitza aquestes matrius per calcular una bossa de models. Aquesta sortida de dades les anomenem "Trainings". Per obtenir aquesta bossa de models, Ichnaea utilitza com a entrada aquestes matrius, uns parametres inicials i les "seasons". Ichnaea retorna una bossa de models que s'utilitza per fer prediccions.

\subsection{Matrius de prediccions}
Ichnaea prediu matrius amb els trainings. Aquestes matrius a difer\`{e}ncia de les originals poden estar incompletes o sense origens. Retornen una taula de valors.



\section{Objectius i estat}
\begin{itemize}
\item Especificar e implementar les interf\'{i}cies de usuari per poder configurar les entrades i execuci\'{o} del software Ichnaea. S'han especificat e implementat quasi totes les interf\'{i}cies de entrada. Falten especificar e implementar les interf\'{i}cies de les entrades predicci\'{o}.
\item Especificar e implementar interficies de usuari per poder veure els resultats de la execuci\'{o} del software Ichnaea. S'han especificat e implementat quasi totes les interf\'{i}cies de sortida. Falten especificar e implementar les interf\'{i}cies de les sortides de predicci\'{o}.
\item Interf\'{i}cies usables, comprensibles i enriquides per tenir una bona experiencia de usuari. La configuraci\'{o} de les matrius \'{e}s la m\'{e}s complexe. S'ha acomplert aquest objectiu mitjançant una API JSON Restful amb la interficies enriquida amb Javascript i Jquery, juntament amb una proto-llibreria propia pel software.
\item Prototipus de llibreria API per en un futur escalar-la i poder integrar el projecte amb qualsevol periferic o tecnologia. Objectiu assolit ja que era requeriment de l'objectiu anterior. Aquesta API es probable que creixi en funcionalitats.
\item Implementar tots aquests objectius en una tecnologia distribuida en xarxa. S'ha aplicat desde el principi tecnologia web.
\item Dissenyar un model de dades flexible que permiti relacionar els objectes per a futures versions de Ichnaea i noves funcionalitats que es poguin desenvolupar mitjançant modificacions o millores de les interficies. S'est\`{a} dissenyant de base un model de dades molt flexible amb un motor de base de dades relacional. Per\'{o} actualment s'est\`{a} limitant la flexibilitat a nivell de interficie per obtenir una millor experi\`{e}ncia de usuari.
\item Integrar la aplicaci\'{o} web amb el projecte "Sistemes de cues per Ichnaea Software" de Miguel Ibero. El projecte est\`{a} integrat per\`{o} encara es troba en desenvolupament. 
\item Estudi de tecnologies per la implementaci\'{o}. El per\'{i}ode de investigaci\'{o} ja s'ha portat a terme. Finalment s'est\'{a} implementant en Symfony2 amb Data Mapper Doctrine per a MySQL i motor de templating TWIG. Totes han sigut tecnologies que s'han apr\'{e}s desde 0.
\item Implementar el prototipus en un entorn real. Actualment el desenvolupament es troba actualitzat al RdLab. Cada dos setmanas es puja la ultima versi\'{o}.
\item Dissenyar un sistema i emprar unes tecnologies escalables i mantenibles. El projecte \'{e}s un prototipus i ofereix m\'{e}s funcionalitat que les que oficialment ofereix Ichnaea. A mesura que Ichnaea ofereixi m\'{e}s funcionalitats, el sistema web s'est\`{a} dissenynat per que sigui mantenible en el futur amb documentacions per a desenvolupadors acurades.
\end{itemize}

