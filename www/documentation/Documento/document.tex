\section{Abast}
L'objectiu del projecte \'{e}s desenvolupar un conjunts de serveis webs per manegar l'algoritme de Backtracking bacteriol\`{o}gic Ichnaea. Actualment Ichnaea es troba en la versi\'{o} 2.0, desenvolupat per Aitor P\'{e}rez P\'{e}rez. La primera versi\'{o} va ser desenvolupada per David Sanchez.\\

La complexitat de les dades i configuracions dels parametres de entrada de Ichnaea, requereixen d'unes interf\'{i}cies i d'un model de dades flexible per poder executar l'algoritme de forma m\'{e}s amigable i comprensible. El prop\`{o}sit del projecte es dissenyar e implementar aquest sistema en un entorn distribu\�{i}t en xarxa.\\

En col.laboraci\'{o} amb Miguel Ibero, que desenvolupa com a Projecte de Final de Carrera un sistema de cues per manegar les execucions de Ichnaea, integrarem una primera versi\'{o}.\\


\section{L'univers Ichnaea}
Per tal de poder entendre els objectius del projecte farem una visi\'{o} global de Ichnaea i els seus elements.\\

\subsection{MST: Microbial Source Tracking}
MST \'{e}s un problema obert en l'actualitat. Consisteix en determinar l'origen biol\`{o}gic dels residus fecals en cossos aquosos. Per fer aix\'{o} es prenen mostres i s'analitzen en un laboratori, i segons els resultats, es decideix si contenen residus fecals d'origen hum\`{a} o de quina familia de animals.\\

Pendre aquesta decisi\'{o} \'{e}s molt dif\'{i}cil. Fins i tot, els microbi\`{o}legs no estan completament segurs de determinar la font les mostres de aigues contaminades. I si poden, a vegades \'{e}s pura intenci\'{o}. Entre els bi\'{o}legs no es possen d'acord de quines variables son rellevants per fer prediccions. La ra\'{o} \'e{s} que les mostres son extretes directament de l'entorn i per aix\'{o} estan diluides i envellides.\\

L'estudi de l'origen de la pol.luci\'{o} en cossos aquasos \'{e}s un problema gran i pot ajudar a assegurar la protecci\'{o} de les poblacions humanes, mostrant una varietat d'enfermetats, especialment en paisos subdesenvolupats.\cite{pfc}\\

\subsection{Ichnaea 2.0}
Ichnaea \'{e}s un software desenvolupat per ajudar a resoldre el problema MST. \'{E}s un eina per llegir matrius de dades(mostres mesurades) i ensamblar diversos conjunts de models. Amb l'ajuda d'aquestes bosses de models, pot llegir noves mostres i fer prediccions dels origens d'aquestes.\cite{pfc}\\

Desde la primera versi\'{o} s'ha refactoritzat el codi i millorat tant el rendiment com els algoritmes.\\

\section{Objectius}
\subsection{Visi\'{o}}
Aquest projecte consisteix en desenvolupar un sistema web dissenyat per executar Ichnaea. Desde el principi ofereix  m\'{e}s funcionalitats que les que dona de base Ichnaea 2.0. L'objectiu \'{e}s poder aprofitar aquest sistema per les futures versions de Ichnaea. Degut a que la execuci\'{o} \'{e}s complexa i de temps elevat, existeix una part del sistema desenvolupada per Miguel Ibero, amb qui es col.labora activament. Aquesta part \'{e}s un sistema de cues per administrar les execucions del software.\\

\subsection{Objectius assolits}
\begin{itemize}
\item Estudi de tecnologies per la implementaci\'{o}. El per\'{i}ode de investigaci\'{o} ja s'ha portat a terme. Finalment s'est\`{a} implementant amb un framework de codi lliure Symfony2 amb Data Mapper Doctrine per a MySQL i motor de templating TWIG. Totes han sigut tecnologies que s'han apr\'{e}s desde 0.

\item Sistema web. Implementar tots aquests objectius en una tecnologia distribuida en xarxa. S'ha aplicat desde el principi tecnologia web usant Symfony2 Framework.

\item Model de dades. Dissenyar un model de dades flexible que permiti relacionar els objectes per a futures versions de Ichnaea i noves funcionalitats que es poguin desenvolupar mitjan\{�}ant modificacions o millores de les interf\'{i}cies. S'est\`{a} dissenyant de base un model de dades flexible amb un motor de base de dades relacional. Per\'{o} actualment s'est\`{a} limitant la flexibilitat a nivell de interf\'{i}cie per obtenir una millor experi\`{e}ncia d'usuari.

\item Interf\'{i}cies d'entrada del sistema. Especificar e implementar les interf\'{i}cies d'usuari per poder configurar les entrades i execuci\'{o} del software Ichnaea. S'han especificat e implementat quasi totes les interf\'{i}cies de entrada. 

\item Interf\'{i}cies de sortida del sistema. Especificar e implementar interficies de usuari per poder veure els resultats de la execuci\'{o} del software Ichnaea. S'han especificat e implementat quasi totes les interf\'{i}cies de sortida. 

\item Disseny de interf\'{i}cies complexes. Interf\'{i}cies usables, comprensibles i enriquides per tenir una bona experi\`{e}ncia d'usuari. La configuraci\'{o} de les matrius \'{e}s la m\'{e}s complexe. S'ha acomplert aquest objectiu mitjan\c{c}ant una API JSON Restful amb les interf\'{i}cies enriquides amb Javascript i Jquery, juntament amb una proto-llibreria pr\`{o}pia pel software.

\item JSON Api RESTful. Prototipus de llibreria API per en un futur escalar-la i poder integrar el projecte amb qualsevol perif\`{e}ric o tecnologia. Objectiu assolit ja que era requeriment de l'objectiu anterior. Aquesta API es probable que creixi en funcionalitats.

\item Dissenyar un sistema i emprar unes tecnologies escalables i mantenibles. El projecte \'{e}s un prototipus i ofereix m\'{e}s funcionalitat que les que oficialment ofereix Ichnaea. A mesura que Ichnaea ofereixi m\'{e}s funcionalitats, el sistema web s'est\`{a} dissenynat per que sigui mantenible en el futur amb documentacions per a desenvolupadors acurades. El fet d'usar Symfony2 ens assegura una bona qualitat de documentaci\'{o} generada per la comunitat i actualment \'{e}s un dels frameworks m\'{e}s reconegut a nivell mundial.

\end{itemize}

\subsection{Objectius per assolir}
\begin{itemize}
\item Interf\'{i}cies d'entrada de prediccions. Falten especificar e implementar les interf\'{i}cies de les entrades predicci\'{o}.
\item Interf\'{i}cies de sortida de prediccions. Falten especificar e implementar les interf\'{i}cies de les sortides de predicci\'{o}.
\item Demostracions i modificacions. Es far\`{a} una demostraci\'{o} al client final i s'establiran dos per\'{i}odes de modificacions.
\item Integraci\'{o} amb cues. Integrar la aplicaci\'{o} web amb el projecte ''Sistemes de cues per Ichnaea Software'' de Miguel Ibero. Aquesta integraci\'{o} t\'{e} dues parts: entrenaments i prediccions. El projecte est\`{a} integrat amb la primera part per\`{o} encara es troba en desenvolupament la segona part. 
\item Prototipus en un entorn producci\'{o}. Actualment el desenvolupament es troba actualitzat al RdLab. Cada dos setmanas es puja la ultima versi\'{o}. Es col.labora amb Miguel Ibero per poder donar una entorn de explotaci\'{o}.
\item Manuals. Manual d'usuari, d'administrador, de desenvolupadors i de administradors de sistema.
\end{itemize}

\section{Planificaci\'{o}}
La planificaci\'{o} per la realitzaci\'{o} de les taquest per assolir \'{e}s la seg\�{u}ent, on cada columna representa una setmana.
\begin{figure}[h!]
  \centering
  \includegraphics[scale=0.5]{gant.png}
  \caption{Planificaci\'{o}}
  \label{fig:placement}
\end{figure}

\begin{thebibliography}{1}
\bibitem{pfc}Aitor P\'{e}rez P\'{e}rez ''ICHNAEA 2.0: a software for microbiology modelling'' pp. 7-34, Feb. 2014
\end{thebibliography}
