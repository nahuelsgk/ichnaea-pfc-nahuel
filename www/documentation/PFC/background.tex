\chapter{Introducci\'{o} a Ichnaea}
\label{cha:background}
En aquest cap\'{i}tol es descriu breument l'univers MST e Ichnaea, necessari per entendre el requeriments. No es dona una visi\'{o} completa del software de com funciona, sino una visi\'{o} global del seu objectiu i quins elements utilitza. 

\section{MST: Microbial Source Tracking}
\label{sec:mst}
MST \'{e}s un problema obert en l'actualitat. Consisteix en determinar l'origen biol\`{o}gic dels residus fecals en cossos aquosos mitjan\c{c}ant l'\'{u}s d'indicadors qu\'{i}mics i microbiol\`{o}gics \cite{paper}. Per fer aix\'{o} es prenen mostres i s'analitzen en un laboratori, i segons els resultats, es decideix si contenen residus fecals d'origen hum\`{a} o de quina familia de animals \cite{pfc}.\\

Pendre aquesta decisi\'{o} \'{e}s molt dif\'{i}cil. Fins i tot, els microbi\`{o}legs no estan completament segurs de determinar la font d'infecci\'{o} de les mostres d'aigues contaminades. La ra\'{o} \'e{s} que les mostres son extretes directament de l'entorn i per aix\'{o} estan diluides i envellides \cite{pfc}.\\

L'estudi de l'origen de la pol.luci\'{o} en cossos aquasos \'{e}s un problema gran i pot ajudar a assegurar la protecci\'{o} de les poblacions humanes, mostrant una varietat d'enfermetats, especialment en paisos subdesenvolupats \cite{pfc}.\\

\section{Ichnaea Software}
Ichnaea \'{e}s un software desenvolupat per ajudar a resoldre el problema MST. \'{E}s un eina per llegir matrius de dades(mostres mesurades) i ensamblar diversos conjunts de models. Amb l'ajuda d'aquestes bosses de models, pot llegir noves mostres i fer prediccions dels origens d'aquestes \cite{pfc}.\\
 
Actualment es troba en la versi\'{o} 2.0. La primera versi\'{o} va ser desenvolupada, com a Master Thesis per David Sanchez, va donar un primer enfoc al problema MST. La segona versi\'{o} ha sigut desenvolupada com a Projecte de Final de Carrera per Aitor P\'{e}rez P\'{e}rez. Ambdues versions han sigut supervisades per Llu\'{i}s Belanche. Desde la primera versi\'{o} s'ha refactoritzat el codi i millorat tant el rendiment com els algoritmes.\\


\section{L'univers d'Ichnaea}
A continuaci\'{o} veurem les entitats amb les que treballa Ichnaea per tal poder donar una visi\'{o} de les dades.

\subsection{Matrius}
\label{cha:backgroud:univers:matrius}
Ichnaea processa inicialment unes matrius on \'e{s} defineixen les mostres de dades extretes, on cada columna representa una variable i cada fila representa una mostra. 

\subsubsection{Variables i conjunts de envelliments}
\label{cha:backgroud:univers:matrius:variables_seasons}
Les variables de Ichnaea tenen asociades uns fitxers. En aquest fitxers s'especifiquen dades mesurades que representen els envelliments de les mostres d'aquestes variables segons la estaci\'{o} de l'any. Aquests fitxers s'agrupen en un conjunt per formar un conjunt d'envelliments.\\

L'objectiu de tenir diferents conjunts d'envelliments \'{e}s tenir agrupats segons les localitzacions. Per exemple, podem tenir la bacteria \textit{Fecal Coliform} amb dos conjunts de envelliments de dos localitzacions diferents del mon. Per exemple, un conjunt de envelliments pot correspondre a mesures fetes a Nairobi i unes altres a Mosc\'{u}. I cada fitxer representra una estaci\'{o} de l'any, ja que segons la estaci\'{o} i la localitzaci\'{o} els envelliments son diferents.\\

\subsection{La matriu i les mostres}
Cada columna d'aquesta matriu, representa una variable de la matriu. Aquesta variable pot ser:\\
\begin{itemize}
\item Una variable que representa una variable de Ichnaea: "variable single"
\item Una variable derivada. Son dos "variables single" relacionadas per una operaci\'{o}.
\item Una variable d'origen, obligatoria per cada mostra. \'{E}s una etiqueta que identifica l'origen de la mostra.
\end{itemize}
Les variables d'origen representa una etiqueta de la mostra per tal de identificar origen de la pol.luci\'{o}. En aquestes matrius els origens son obligatoris i cada columna ha de tenir un valor definit.\\

\subsection{Entrenaments}
Ichnaea processa aquestes matrius amb un conjunt d'envelliments per calcular una bossa de models. Aquesta proc\'{e}s s'anomena entrenaments. \\
Aquestes bosses de m\`{o}dels resultants s'utilitzen per fer prediccions.

\subsection{Matrius de prediccions}
Les dades que necessita Ichnaea per fer prediccions son un conjunt de noves mostres en forma de matriu. A partir d'un entrenament, pot fer prediccions d'origens de contaminaci\'{o}. Aquestes matrius son molt similars descrites a la secci\'{o} \ref{cha:backgroud:univers:matrius}. La difer\'{e}ncia \'{e}s que les mostres no han de ser completes. Per exemple, les mostres no tenen perque tenir un origen o poden valors per variables sense definir.

\subsection{Sistema de cues}
Ichnaea Software requereix d'un cost alt de proc\'{e}s tant en rendiment com en temps. Aquest projecte s'ha desenvolupat en paral.lel amb el Projecte de Final de Carrera de Miguel Ibero que implementa un sistema de cues d'execuci\'{o} de Ichnaea. Aquest projecte s'ha de integrar amb aquest sistema seguint els requeriments del projecte ''Sistema de cues per a Ichnaea Software''.