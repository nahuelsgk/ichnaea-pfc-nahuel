\chapter*{Manual d'usuari}
\label{cha:userguide}

\section{Casa de l'usuari}
\label{sec:home}
Per accedir al home pots:
\begin{itemize}
\item accedir nom\'{e}s loguejan-te 
\item desde "My matrixs" 
\item desde "Trainings > My Trainings" o al logo de "Ichnaea".
\end{itemize}

En el dashboard es poden veure dues seccions. Les matrius(part esquerra) i els trainings(part dreta) creats per l'usuari. 
Nota: Actualment no es poden compartir propietat de matrius. Es a dir, una sola persona(el propietari/creador) solament pot editar la matriu. Per poder solventar aquest trau, existeix la possibilitat de clonar una matriu. Mirar \ref{sec:clone_matrix}.
\begin{figure}[h!]
  \centering
  \includegraphics[scale=0.2]{img/userguide/dashboard_complete_trainings.png}
  \caption{Casa de l'usuari}
  \label{fig:placement}
\end{figure}
 
\subsection{Llistat de les meves matrius}
Al llistat de la esquerra pots veure les matrius que ets propietari. Amb la icona de edici\´{o} pots anar a la pantalla de configuraci\´{o}. Per configurar la matriu, mirar \ref{sec:configure_matrix}.

\subsection{Llistat dels meus trainings}
Al llistat de la dreta pots veure els trainings que has creat on::
\begin{itemize}
\item Operacions
 \begin{itemize}
 \item La icona de l'ull es per anar a la pantalla de visualitzaci\´{o} del training.
 \item La icona de la creu, \´{e}s per esborrar el training.
 \item La icona de la quadricula, \'{e}s per crear prediccions i matrius de prediccions.
 \end{itemize}
\item Nom del training
\item Nom de la matriu entrenada
\item Progres: actualment Ichnaea no retorna estat del proces- Solament ens diu si ha acabat o no.  Per tant els \'{u}nics valors son 0.00 i 1.00.
\item Status. Actualment \´{e}s 0 o 1. Es a dir, acabada de entrenar o no. Ichnaea encara no dona status parcials de quan temps li queda per acabar o quan porta.
\end{itemize}
\subsubsection*{Estat del trainings}
En la figura es contempla els estats possibles:
\begin{itemize}
\item Color verd: training sense errors i predectible
\item Color gris: training actualment corrent
\item Color salm\'{o}: training no enviat per algun problema amb la cua o amb errors.
\end{itemize}

\section{Crear una matriu desde un csv}
\label{sec:create_matrix}
Desde el menu superior "IchnaeaData > New matrix", es pot pujat una nova matriu en format csv. El format csv es compatible amb les programaris de ofim\`{a}tica m\´{e}s habituals como Microsoft Excel o Libreoffice.
El format de la matriu \´{e}s important que sigui el seg\¨{u}ent.
\begin{center}
    \begin{tabular}{ | l | l | l | p{5cm} |}
    \hline
    Cel.la bu\¨{i}da & Alias de la columna & .... & ORIGIN \\ \hline
    Nom de la sample & Valor de la sample  & .... & Origen de la sample \\ \hline
    S01-10-20        & 0,000145            & .... & Human \\ \hline
    \hline
    \end{tabular}
\end{center}
On:
\begin{itemize}
\item Alias de la columna: \´{e}s un nom qualsevol per identificar la columna
\item Valor de la sample: \´{e}s el valor de la mostra per la columna(variable)
\item Nom de la sample: \'{e}s un identificador de la mostra
\item Origen de la sample: \'{e}s una cadena de caracters que especifica l'origen de la mostra. Solament es distingir\'{a} si a les capçaleres a la ultima columna s'expecifica la paraula ORIGIN.
\end{itemize}
En la pantalla, es pot seleccionar un fitxer csv i pujar'ho. Seguidament, es podr\`{a} establir la relaci\´{o} de la variable real de la columna i quin conjunt de season per defecte usa. Mirar \ref{sec:configure_matrix}.

\section{Configurar una matriu}
\label{sec:configure_matrix}
Per accedir a configurar una matriu, has d'anar a la teva pantalla de inici(mirar \ref{sec:home}). 
Desde la interficie de configuraci\´{o} es pot configurar:
\begin{itemize}
\item Donar un alias a la columna
\item Asociar una columna a una variable
\item Seleccionar un conjunt de seasons de la variable
\item Donar nom a un sample
\item Donar una data a un sample
\item Donar un origen a un sample
\item Visualitzar missatges de validacions i notificacions
\item Donar acces als usuaris per que puguin crear trainings.
\end{itemize}
\begin{figure}[h!]
  \centering
  \includegraphics[scale=0.2]{img/userguide/matrix_configure.png}
  \caption{Interficie de configuraci\'{o} de matrius}
  \label{fig:configure_matrix}
\end{figure}

\subsection{Alias de una columna} 
A la secci\'{o} de les capçaleres, a la icona del llapis, es pot especificar un alias a la columna. Si es prem "Enter" o es canvia el focus, s'activa el bot\'{o} de salvaguardat.
\subsection{Especificar una columna a una variable i la season set}
A la secci\'{o} de les capçaleres, a la icona de la clau anglesa, es pot seleccionar la variable del sistema. Autom\'{a}ticament, a la llista de dalt, es carrega la llista de "Seasons Set". Quan es selecciona un dels dos llistats, s'activa el bot\'{o} de salvaguardat. No \'{e}s obligatori donar-li una variable o una "Season Set".
\subsection{Cambiar la visualitzaci\'{o}}
A la secci\'{o} de configuraci\'{o}, es pot cambiar la visibilitat. Si la matriu \'{e}s invisible, els usuaris no poden crear trainings. Per guardar els canvis, s'ha de pitjar el but\'{o} "Save configuration".
\subsection{Visualitzar missatges}
Existeixen diverses restriccions i missatges:
\begin{itemize}
\item Notificaci\'{o} de visibilitat: una matriu visible es entrenable. 
\item Notificaci\'{o} de matriu amb trainings creats. Una modificaci\'{o} crea una incoherencia amb aquests trainings ja que no ser\'{a} la mateixa matriu.
\item Notificaci\'{o} origins. Les mostres necessiten obligatoriament uns origins.
\end{itemize}

\section{Clonar una matriu}
\label{sec:clone_matrix}
Desde el llistat menu "Ichnaea Data > View Matrix", podem accedir al llistat de variables del sistema. Amb la icona etiquetada com "Clone the matrix", podem clonar una matriu sencera configurada. No es copien els trainings.
\begin{figure}[h!]
  \centering
  \includegraphics[scale=0.2]{img/userguide/clone_matrix.png}
  \caption{Llistat de matrius}
  \label{fig:placement}
\end{figure}
Fent click a la icona de reload, anem al formulari que suggereix un nom per identificar-la.
\begin{figure}[h!]
  \centering
  \includegraphics[scale=0.2]{img/userguide/clone_matrix-2.png}
  \caption{Llistat de matrius}
  \label{fig:placement}
\end{figure}
Acceptant, es clona la matriu i anem a la interficie de configuraci\´{o}. Mirar \ref{sec:configure_matrix}.

\section{Crear un conjunt de season}
Pendent de escriure.

\section{Modificar o afegir m\´{e}s season a una variable}
Pendent de escriure.

\section{Crear un training}
Per crear un training s'ha de accedir al menu superior "Trainings > Create a training". Desde el llistat de matrius del sistema, amb la icona de la "carretera", es pot accedir al formulari de creaci\´{o} de trainings.
\begin{figure}[h!]
  \centering
  \includegraphics[scale=0.2]{img/userguide/create_.png}
  \caption{Llistat de matrius}
  \label{fig:placement}
\end{figure}
\begin{itemize}
\item Part esquerra superior: El training cont\´{e} un nom i una descripci\´{o}. Es poden seleccionar quines columnes vols entrenar de la matriu. 
\item Part dreta superior: Desplegable per seleccionar un dels origens disponibles.  El origen-versus, es un llistat de la variable origen de la matriu. Si es selecciona el valor "All versus all", el training ser\´{a} tots contra tots. Si \´{e}s selecciona un origen concret, el training es far\´{a} aquest origen contra els altres. Actualment Ichnaea no suporta aquesta part per\'{o} en el futur est\`{a} planificat que ho far\`{a}.
\item Selecci\'{o} de columnes. Selecci\'{o} de columnes que vols entrenar.
\end{itemize}

Si la creaci\´{o} \´e{s} correcte, les dades s'enviaran a la cua de procesos i la aplicaci\´{o} es redirigir\´{a} la pantalla de visualitzaci\´{o} de trainings.

\subsection{Simular un training de la matriu Cyprus}
Actualment la aplicaci\'{o} Ichnaea i el sistema de cues no esta implantat. Tenim la opci\'{o} de tenir una matriu entrenada en un altre plataforma per poder fer proves amb les interficies de prediccions. PENDENT DE CONFIRMAR REQUERIMENT.

\section{Visualitzar un training}
Desde la casa de l'usuari(mirar \ref{sec:home}), es pot veure els teus trainings i en quin estadi es troben. Amb la icona "ull", pots accedir a visualitzar la informaci\´{o} del training.

\begin{itemize}
\end{itemize}

\subsection{Problem\'{a}tiques de la creaci\'{o} de trainings}
\subsubsection*{Error en el enviament}
\begin{figure}[h!]
  \centering
  \includegraphics[scale=0.2]{img/userguide/view_training_pending.png}
  \caption{Training que es pot enviar a la cua}
  \label{fig:placement}
\end{figure}

Actualment Ichnaea Software i el sistema de cues no esta implantat. Per defecte, la creaci\'{o} donar\'{a} error. Per tant, es pot utilitzar la simulaci\'{o} de trainings.
A banda d'aix\'{o}, tenim predifinits un conjunt de situacions que a continuaci\'{o} descrivim.

\section{Crear una matriu de predicci\'{o}}
Desde la casa de l'usuari, es pot veure els teus trainings i en quin estadi es troben. Amb la icona "Quadradets"

\section{Crear una predicci\'{o}}
Desde la casa de l'usuari(mirar \ref{sec:home}) es pot crear una predicci\'{o} d'un training. Seleccionant la icona "quadricules" de un trainig correcte(en color verd), es pot crear un matriu de predicci\'{o}.
\begin{figure}[h!]
  \centering
  \includegraphics[scale=0.2]{img/userguide/create_prediction.png}
  \caption{Exemple de creaci\'{o} de predicci\'{o}}
  \label{fig:placement}
\end{figure}
A part superior \'{e}s pot seleccionar un fitxer per pujar la matriu per predir.
A la part inferior es pot veure les columnes que el training t\'{e} seleccionades. La matrius en format csv ha de tenir el format indicat per la part inferior. En breu es podr\'{a} descarregar una template per tenir el template i poder simplement omplir els valors:
\begin{center}
    \begin{tabular}{ | l | l | l | p{5cm} |}
    \hline
    Cel.la bu\¨{i}da & Columna del training 0          & Columna del training 1              & .... & Columna del training n       & ORIGIN \\ \hline
    Nom de la sample & Valor de la sample(opcional)    & Valor de la sample(opcional)        & .....& Valor de la sample(opcional) & Origen de la sample(opcional) \\ \hline
    S01-10-20        & 0,000145                        &                                     & .... & <10                                                 & Human \\ \hline
    \hline
    \end{tabular}
\end{center}
Seguidament es pot visualitzar

\section{Llistar les meves prediccions}
Cas d'us pendent d'especificar

\section{Visualitzar una matriu de predicci\'{o}}
Cas d'us pendent d'especificar.

\subsection{Actualitzar una matriu de predicci\'{o}}

\subsection{Executar una predicci\'{o} de una matriu de predicci\'{o}}


