\chapter{Disseny}
\label{cha:dessign}

\section{Introduccio}
En aquest capitol veurem els patrons de dissenys emprats i les tecnologies implementades.

\section{Patro de disseny}
S'ha emprat el framework Symfony. El patr\´{o} usat \´{e}s:
  \begin{itemize}
  \item Model-Vista-Controlador 
  \item Front Crontroller
  \item Capa de Servei amb injecci\´{o} de depend\`{e}ncies
  \item mapejat de dades
  \item View template
  \item Interficies enriquides amb servei webs
  \end{itemize}

\section{Symfony}
Symfony2 \´{e}s un framework que implementa Front Controller amb injecci\´{o} de depencies a la capa de serveis. Les entitats(model de dades) s'han implementat directament amb l'ORM doctrine contra una base de dades MySQL.
Arquitectonicament, Symfony2 estructura el codi en Bundle. Son unitat funcionals independents de recursos. En aquest cas els "bundles" son un conjunt de serveis, entitats i recursos html. Els bundles implementats son
\begin{itemize}
\item{User}
\item{Matrix}
\item{Training}
\item{ApiBundle}
\end{itemize}

\section{Recursos}
La estructura de recursos \´{e}s la seg\¨{u}ent:

matrix/{id}/training

\section{Servei web}
S'ha emprat un API JSON Restful per enriquir les interficies. S'ha emprat aquesta tecnologia per la escalabilitat que aporta i perque en un futur es pogui aprofitar el desenvolupament d'aquesta.
Les operacions, els recursos i el parametres son:

\section{Capa de serveis}

\section{Integraci\´{o} amb el sistema de cues RabbitMQP}




