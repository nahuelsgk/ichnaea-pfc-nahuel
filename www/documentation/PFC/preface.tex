
\chapter*{Introducci\'{o}}

\section{Abast}
L'objectiu del projecte es desenvolupar un conjunts de serveis webs per manegar l'algoritme de Backtracking bacteriol\`{o}gic Ichnaea. Actualment Ichnaea es troba en la versi\´{o} 2.0, desenvolupat per Aitor APELLIDO. La primera versi\´{o} va ser desenvolupada per David Sanchez. La complexitat de les entrades i configuracions dels parametres de entrada de  Ichnaea, requereixen de una interficies i d'un model de dades per poder executar l'algoritme. El proposit del projecte es dissenyar e implementar aquest sistema en un entorn distribu\¨{i}t en xarxa. 

El document seg\¨{u}ent cont\´{e}:
\begin{itemize}
\item En el capitol \reg{cha:glossary} s'especifica el vocabulari emprat i els conceptes propis del projecte. Necessari per entendre els requeriements
\item En el capitol \ref{cha:especification} es descriu els requeriments i les funcionalitats que es requereixen.
\item En el capitol \ref{cha:dessign} es descriu els patrons de dissenys i les solucions emprades per implementar el projecte
\item En el capitol \ref{cha:userguide} es dona un breu manual de usuari per fer servir la aplicaci\´{o}.
\item En el capitol \ref{cha:userguide} es dona un breu manual de usuari per fer servir la aplicaci\´{o}.

\end{itemize}

\section{Objectius}
Els objectiu principals del projecte son:
\begin{itemize}
\item Especificar e implementar interficies de usuari per poder configurar les entrades i execuci\'{o} del software Ichnaea.
\item Especificar e implementar interficies de usuari per poder veure els resultats de la execuci\'{o} del software Ichneae.
\item Interficies usables, comprensibles i enriquides per tenir una bona experiencia de usuari.
\item Prototipus de llibreria API per en un futur escalar-la i poder integrar el projecte amb qualsevol periferic o tecnologia.
\item Implementar tots aquests objectius en una tecnologia distribuida i web.
\item Dissenyar un model de dades flexible que permiti relacionar els objectes per a futures versions de Ichnaea i noves funcionalitats que es poguin desenvolupar mitjançant modificacions o millores de les interficies.
\item Integrar la aplicaci\'{o} web amb el projecte "NOM DEL PROJECTE" de Miguel Ibero.
\end{itemize}

