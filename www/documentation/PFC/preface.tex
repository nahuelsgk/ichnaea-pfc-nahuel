\chapter{Prefaci}
\label{cha:prefaci}
\section{Abast}
\label{sec:abast}

L'objectiu del projecte es desenvolupar un conjunts de serveis web per manegar l'algoritme de Backtracking bacteriol\`{o}gic Ichnaea. Actualment Ichnaea es troba en la versi\'{o} 2.0, desenvolupat per Aitor Perez P\'{e}erez P\'{e}erez. La primera versi\´{o} va ser desenvolupada per David Sanchez. \\

La complexitat de les entrades i configuracions dels parametres de entrada de Ichnaea, requereixen de unes interf\'{i}cies enriquides i d'un model de dades flexible per poder executar l'algoritme. Per un altre banda i com a segon objectiu principal \'{e}s integrar el Projecte de Final de Carrera de Miguel Ibero "Sistema de cues per Ichnaea Software", on s'est\`{a} dissenyant i desenvolupant un sistema de cues per manegar les execucions ja que requereixen d'un temps elevat de proc\'{e}s. Com a tercer objectiu, s'ha de dissenyar e implementar aquest sistema en un entorn distribu\�{i}t en xarxa. \\

El document seguent conte:
\begin{itemize}
\item En el cap\'{i}tol \ref{cha:background} es fa una petita introducci\'{o} a Ichnaea i al problema MST. 
\item En el cap\'{i}tol \ref{cha:specification} s'especifica els requeriments, els casos d'ussos i el m\'{o}del de dades.
\item En el cap\'{i}tol \ref{cha:dessign} s'especifica el disseny emprat, les tecnologies i les interf\'{i}cies m\'e{s} complexes.
\item En el cap\'{i}tol \ref{cha:conclussions} s'especifica l'estudi econ\`{o}mic del projecte i les possibles millores.
\item En l'annex \ref{cha:userguide} es dona un petit manual d'usuari.
\item En l'annex \ref{cha:adminguide} es dona un petit manual de administrador.
\end{itemize}

\section{Objectius}
\label{sec:objetius}

Els objectiu principals del projecte son:\\

\begin{itemize}
\item Especificar e implementar les interf\'{i}cies de usuari per poder configurar les entrades i execuci\'{o} del software Ichnaea.
\item Especificar e implementar interficies de usuari per poder veure els resultats de la execuci\'{o} del software Ichnaea. 
\item Interf\'{i}cies usables, comprensibles i enriquides per tenir una bona experiencia de usuari. 
\item Prototipus de llibreria API per en un futur escalar-la i poder integrar el projecte amb qualsevol periferic o tecnologia. 
\item Implementar tots aquests objectius en una tecnologia distribuida en xarxa. 
\item Dissenyar un model de dades flexible que permiti relacionar els objectes per a futures versions de Ichnaea i noves funcionalitats que es poguin desenvolupar mitjançant modificacions o millores de les interficies. 
\item Integrar la aplicaci\'{o} web amb el projecte "Sistema de cues per Ichnaea Software" de Miguel Ibero.
\end{itemize}
