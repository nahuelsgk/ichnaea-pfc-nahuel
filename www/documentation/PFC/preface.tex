\chapter*{Introducci\'{o}}

\section{Abast}
L'objectiu del projecte es desenvolupar un conjunts de serveis webs per manegar l'algoritme de Backtracking bacteriol\`{o}gic Ichnaea. Actualment Ichnaea es troba en la versi\´{o} 2.0, desenvolupat per Aitor Perez P\'{e}erez P\'{e}erez. La primera versi\´{o} va ser desenvolupada per David Sanchez. \\

La complexitat de les entrades i configuracions dels parametres de entrada de Ichnaea, requereixen de unes interf\'{i}cies enriquides i d'un model de dades flexible per poder executar l'algoritme. Per un altre banda i segon objectiu principal, integrar el Projecte de Final de Carrera de Miguel Ibero "TITULO", on s'est\`{a} dissenyant un sistema de cues per manegar les execucions ja que requereixen d'un temps elevat de proc\'{e}s. Com a tercer objectiu, s'ha de dissenyar e implementar aquest sistema en un entorn distribu\¨{i}t en xarxa. \\

El document seguent conte:
\begin{itemize}
\item En el cap\'{i}tol \ref{cha:background} s'especifica els conceptes propis del projecte. Necessari per entendre els requeriments
\item En el cap\'{i}tol \ref{cha:background} es fa una petita introducci\'{o} als conceptes i entitats que Ichnaea tracta, necessari per entendre tota les funcionalitat i requeriments.
\item En el cap\'{i}tol \ref{cha:specification} s'especifiquen el requeriments, les funcionalitats necessaries i els casos de usos.
\item En el cap\'{i}tol \ref{cha:dessign} s'especifica el disseny emprat, les tecnologies i les interf\'{i}cies m\'e{s} complexes
\item En el cap\'{i}tol \ref{cha:userguide} es dona un petit manual d'usuari
\item En el cap\'{i}tol \ref{cha:adminguide} es dona un petit manual de administrador
\item En el cap\'{i}tol \ref{cha:developguide} es dona un petit manual per el desenvolupador.
\end{itemize}

\section{Objectius}
Els objectiu principals del projecte son:

\begin{itemize}
\item Especificar e implementar les interf\'{i}cies de usuari per poder configurar les entrades i execuci\'{o} del software Ichnaea.
\item Especificar e implementar interficies de usuari per poder veure els resultats de la execuci\'{o} del software Ichnaea. 
\item Interf\'{i}cies usables, comprensibles i enriquides per tenir una bona experiencia de usuari. 
\item Prototipus de llibreria API per en un futur escalar-la i poder integrar el projecte amb qualsevol periferic o tecnologia. 
\item Implementar tots aquests objectius en una tecnologia distribuida en xarxa. 
\item Dissenyar un model de dades flexible que permiti relacionar els objectes per a futures versions de Ichnaea i noves funcionalitats que es poguin desenvolupar mitjançant modificacions o millores de les interficies. 
\item Integrar la aplicaci\'{o} web amb el projecte "NOM DEL PROJECTE" de Miguel Ibero. El projecte est\`{a} integrat per\`{o} encara es troba en desenvolupament. 
\end{itemize}
