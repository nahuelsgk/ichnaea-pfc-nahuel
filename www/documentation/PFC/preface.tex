\chapter{Prefaci}
\label{cha:prefaci}

\section{Introducció}
\label{sec:abast}
Aquest projecte desenvolupa el disseny i la implementació d'un sistema per manegar l'algoritme de Backtracking bacteriològic Ichnaea. El principal objectiu del \textit{software} Ichnaea \'{e}s determinar l'origen de la pol·lució fecal en cossos aquosos mitjançant la generació de bosses de models i l'anàlisis de mostres. Aquest problema es conegut com MST(Microbial Source Tracking).\\

En aquest document veurem una breu descripció de l'univers d'Ichnaea, el disseny i la implementació d'aquest sistema i les conclusions arribades amb la realització d'aquest projecte.

\section{Motivació}
La motivació d'aquest projecte es evolucionar el \textit{software} Ichnaea dotant a l'algoritme d'un sistema i d'unes interfícies per administrar-lo i executar-lo. La complexitat tant de les entrades i de les configuracions dels paràmetres de Ichnaea com de les sortides, fa que es requereixi la realització d'aquest projecte.\\

Actualment Ichnaea es troba en la versi\'{o} 2.0, desenvolupat per Aitor P\'{e}rez P\'{e}rez. La primera versi\'{o} va ser desenvolupada com a tesis per David Sànchez. En la actualitat no existeix cap algoritme de Backtracking Bacteriologic ni cap sistema similar destinat aquests problema.\\

Paral·lelament a aquest PFC, Miguel Ibero desenvolupa el PFC "Sistema de cues per a Ichnaea". Ambdós, juntament amb les futures versions de Ichnaea, s'integren i formen la evolució de Ichnaea com a un sistema complexe.

\section{Objectius}
\label{sec:objetius}
Els objectius principals del projecte son dos. En primer lloc, dotar un sistema multi capa robust i distribuït en xarxa amb la capacitat de tenir  interfícies enriquides i d'un model de dades flexible per poder configurar i executar l'algoritme. En segon lloc integrar el Projecte de Final de Carrera de Miguel Ibero "Sistema de cues per Ichnaea Software", on s'est\`{a} dissenyant i desenvolupant un sistema de cues per manegar les execucions\\

Per assolir els objectius principals he desenvolupat els següents objectius específics:
\begin{itemize}
\item Estudiar l'algoritme Ichnaea i les seves entitats per dissenyar un model de dades.
\item Especificar e implementar les interf\'{i}cies de usuari per poder configurar les entrades i execucions de Ichnaea.
\item Especificar e implementar interf\'{i}cies d'usuari per poder veure els resultats de la execuci\'{o} del software Ichnaea. 
\item Interf\'{i}cies usables, comprensibles i enriquides per tenir una bona experiència de usuari. 
\item Dissenyar e implementar una llibreria API per integrar amb futurs sistemes o tecnologies.
\item Implementar tots aquests objectius en una tecnologia distribuïda en xarxa. 
\item Dissenyar un model de dades flexible que permeti evolucionar el sistema per a futures versions de Ichnaea.
\item Integrar el sistema amb el projecte "Sistema de cues per Ichnaea Software" de Miguel Ibero.
\end{itemize}

\section{Estructura del document}
El document s'estructura de la seguent manera:
\begin{itemize}
\item Al cap\'{i}tol \ref{cha:background} es fa una petita introducci\'{o} a Ichnaea i al problema MST. 
\item Al cap\'{i}tol \ref{cha:specification} s'especifica els requeriments, els casos d'usos i el m\'{o}del de dades.
\item Al cap\'{i}tol \ref{cha:dessign} s'especifica el disseny de l'aplicació.
\item Al capítol \ref{cha:implementation} es descriuen les tecnologies i la implementació del projecte.
\item Al capítol \ref{cha:tests} es descriuen les proves i les dificultats trobades.
\item Al cap\'{i}tol \ref{cha:conclussions} es descriu la metodologia, la evolució i l'estudi econòmic del projecte i les possibles millores.
\item En l'annex \ref{cha:userguide} es dona un petit manual d'usuari.
\item En l'annex \ref{cha:adminguide} es dona un petit manual de administrador.
\end{itemize}

