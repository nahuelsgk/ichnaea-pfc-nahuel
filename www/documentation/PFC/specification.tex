\chapter{An\`{a}lisis i Especificaci\'{o}}
\label{cha:specification}
En aquest cap\'{i}tol es descriuen els requeriments, les operatives que es necessiten i el model de dades.

\section{An\`{a}lisis de Requeriments}

\subsection{Requeriments funcionals}
\subsubsection{Administraci\'{o} d'usuaris}
La aplicaci\'{o} ha de estar protegida i autoritzada pels usuaris. Els usuaris autenticats han de tenir permisos i  pertanyer a grups amb rols autoritzats per fer certes accions. \'{E}s una aplicaci\'{o} distribuida per tant s'ha de dissenyar un sistema que permeti:
\begin{itemize}
\item Crear comptes d'usuari.
\item Atendre peticions de resetejar contrasenyes.
\item Enviar mails de confirmacions de accions.
\item Canviar permisos a usuaris.
\end{itemize}

\subsubsection{Administraci\'{o} de variables de sistema}
La aplicaci\'{o} ha de gestionar les variables que Ichnaea al sistema per poder utilitzar-les amb el software. Les variables han de tenir asociades conjunts de fitxers\ref{cha:backgroud:univers:matrius:variables_seasons}). La aplicaci\'{o} ha de poder gestionar les variables, els continguts dels fitxers i les associacions entre les variables i els conjunts de fitxers.

\subsubsection{Administraci\'{o} de matrius}
La aplicaci\'{o} ha de gestionar i configurar matrius de dades. Per la creaci\'{o} de matrius ha de poder llegir una fulla de c\`{a}lcul i crear un model de dades que representi una matriu a partir de les dades proporcionades.\\

La aplicaci\'{o} ha de permetre la configuraci\'{o} de matrius\ref{cha:backgroud:univers:matrius}:
\begin{itemize}
\item Gestionar matrius: crear, actualitzar, esborrrar i configurar-les.
\item Configurar les columnes d'una matriu com una variable i un conjunt de fitxers en cas de que tinguin m\'{e}s.
\item Configurar l'origen d'una mostra de la matriu.
\item Configurar la data d'una mostra de la matriu.
\item Configurar o actualitzar dades b\`{a}siques d'una matriu.
\end{itemize}

\subsubsection{Administraci\'{o} de entrenaments}
El sistema ha de gestionar i crear entrenaments i enviar-ho contra una cua d'execuci\'{o} d'Ichnaea:
\begin{itemize}
	\item Gestionar entrenaments: crear, configurar i esborrar.
	\item Enviar a processar-los a la cua de processos.
	\item Llegir e interpretar el estat del proc\'{e}s.
	\item Guardar els resultats dels entrenaments.
	\item Gestionar les sortides de les execucions.
\end{itemize}

\subsubsection{Administraci\'{o} de matrius de predicci\'{o}}
El sistema ha de gestionar i crear noves matrius de prediccions i enviar-ho contra una cua d'execuci\'{o}.
\begin{itemize}
	\item Gestionar les matrius de prediccions.
	\item Enviar a processar-los a la cua de processos.
	\item Llegir e interpretar el estat del proc\'{e}s.
	\item Llegir els resultats de les prediccions.
	\item Gestionar les sortides de les execucions.
\end{itemize}

\subsection{Requeriments no funcionals}
Els requeriments no funcionals son:
\begin{itemize}
\item Bon rendiment
\item Escalabilitat
\item Mantenible
\item Flexibilitat
\item Usabilitat
\end{itemize}

\section{M\`{o}del de Casos d'us}

\subsection{Actors}
Els usuaris del sistema son:
\begin{figure}[h]
  \centering
  \includegraphics[scale=0.5]{img/specification/Actors.png}
  \caption{Actors}
  \label{specification:actors}
\end{figure}
\begin{itemize}
\item Usuari an\`{o}nim: usuari sense compte al sistema.
\item Usuari registrat: usuari amb compte al sistema.
\begin{itemize}
\item Propietari d'una matriu: usuari que ha creat una matriu.
\item Propietari d'un entrenament: usuari que ha creat un entrenament.
\item Propietari d'una predicci\'{o}: usuair que ha creat una matriu de predicci\'{o}.
\end{itemize}
\item Usuari administrador: usuari amb permissos administratius.
\item Cua: usuari l\'{o}gic(no \'{e}s una persona f\'{i}sica) dels sistema que gestiona les execucions.
\item Consumidor: usuari l\'{o}gic(no \'{e}s una persona f\'{i}sica) del sistema que gestiona les sortides de les execucions.
\item Sistema: sistema que rep les peticions i gestiona les sortides.
\end{itemize}
\subsection{Diagrama dels casos d'\'{u}s}
A la figura \ref{fig:usecases} es pot veure la el diagrama de casos d'\'{u}s.
\begin{figure}[h]
  \centering
  \includegraphics[scale=0.2]{img/specification/UsesCases.png}
  \caption{Diagrama de casos d'\'{u}s}
  \label{fig:usecases}
\end{figure}

\subsection{Especificaci\'{o} dels casos d'\'{u}s}
En la documentaci\'{o} s'utilitzar\`{a} la seg\�{u}ent estructura per definir els casos d'\'{u}s:\\
\begin{usecase}
\addtitle{Identificador}{Nom cas d'us}
\addfield{Actors:}{Llista de actors}
\addscenario{Curs t\'{i}pic d'esdeveniments:}{
	\item Esdeveniment
	\item Esdeveniment
	\item ...
}
\addscenario{Cursos alternatius:}{
\item Esdeveniment Alternatiu
}
\end{usecase}


\subsubsection{Crear un usuari}
\begin{usecase}
\addtitle{Usuari001}{Crear un usuari}
\addfield{Actors:}{Anonim}
\addscenario{Curs t\'{i}pic d'esdeveniments:}{
	\item Usuari accedeix al formulari de registraci\'{o}. L'usuari introdueix un nom d'usuari, un correu electr\`{o}nic i una contrasenya per duplicat
	\item El sistema envia al usuari una confirmaci\'{o} via correu electr\'{o}nic amb un enlla� de confirmaci\'{o} i crea un compte no validada.
	\item L'usuari accedeix mitjan�ant l'enlla� de confirmaci\'{o}.
	\item El sistema comprova que \'{e}s un enlla� de confirmaci\'{o} v\`{a}lid d'aquest usuari i activa la compte. L'usuari ja est\'{a} autenticat al sistema com un usuari i ja \'{e}s un usuari del sistema.
}
\addscenario{Cursos alternatius:}{
\item[3] El sistema valida que no existeixi un usuari amb aquesta compte de correu, que el correu sigui v\`{a}lid. Sino es correcte li informa a l'usuari al mateix formulari.
}
\end{usecase}

\subsubsection{Canviar un usuari de grup}
\begin{usecase}
\addtitle{Usuari002}{Canviar un usuari de grup}
\addfield{Actors:}{Administrador}
\addscenario{Curs t\'{i}pic d'esdeveniments:}{
	\item L'administrador llista tots els usuaris del sistema i selecciona un.
	\item L'administrador veu el formulari de edici\'{o} de permisos.
	\item L'administrador selecciona el nou perm\'{s} i salva el perfil.
	\item El sistema guarda el nou perm\'{i}s
}
\end{usecase}

\subsubsection{Crear una variable}
\begin{usecase}
\addtitle{Variable001}{Crear una variable}
\addfield{Actors:}{Usuari administrador}
\addscenario{Curs t\'{i}pic d'esdeveniments:}{
	\item L'usuari visualitza el formulari on pot donar un identificador que ha de ser \'{u}nic en el sistema i una descripci\'{o} .
	\item El sistema crear la variable.
}
\end{usecase}

\subsubsection{Actualitzar una variable}
\begin{usecase}
\addtitle{Variable002}{Actualitzar una variable}
\addfield{Actors:}{Usuari administrador}
\addscenario{Curs t\'{i}pic d'esdeveniments:}{
    \item L'admininistrador seleccion un variable d'un llistat de variables.
    \item El sistema li mostra un formulari d'edici\'{o} on pot veure els conjunts dels fitxers i pot actualitzar la descripci\'{o}.
    \item L'usuari modifica la descripci\'{o} i salva els canvis.
    \item El sistema guarda les modificacions.
}
\end{usecase}

\subsubsection{Crear un conjunt de fitxers d'envelliment per una variable}
\begin{usecase}
\addtitle{Variable003}{Crear un conjunt de fitxers d'envelliment per una variable}
\addfield{Actors:}{Usuari registrat}
\addscenario{Curs t\'{i}pic d'esdeveniments:}{
	\item L'usuari selecciona una variable d'un llistat de variables.
	\item El sistema li mostra un formulari d'edici\'{o} de la variable amb un enlla� a un formulari de creaci\'{o}.
	\item L'usuari accedeix a un formulari de creaci\'{o} d'un conjunt de fitxers.
	\item El sistema li mostra un formulari de creaci\'{o}.
	\item L'usuari pot donar un nom i seleccionar 0, 1 o 2 fitxers, on cada fitxer pot ser configurat com:
	\begin{itemize}
		\item a unic per tot l'any
		\item com estiu
		\item com hivern
		\item com tardor
		\item com estiu
	\end{itemize}	
	\item L'usuari salva els canvis.
	\item El sistema guarda els canvis.
}
\end{usecase}

\subsubsection{Actualitzar un conjunt de fitxers d'envelliments}
\begin{usecase}
\addtitle{Variable004}{Actualitzar un conjunt de fitxers d'envelliments}
\addfield{Actors:}{Usuari registrat}
\addscenario{Curs t\'{i}pic d'esdeveniments:}{
    \item L'usuari selecciona una variable d'un llistat de variables.
    \item El sistema li mostra un formulari d'edici\'{o} on pot editar el nom.
    \item L'usuari selecciona un conjunt de fitxers.
    \item El sistema li mostra un formulari on pot actualitzar un conjunt de fitxers.
    \item L'usuari actualiza les dades i guarda els canvis.
    \item El sistema guarda els canvis. 
}
\end{usecase}

\subsubsection{Esborrar un conjunt d'envelliments d'una variable}
\begin{usecase}
\addtitle{Variable005}{Esborrar un conjunt d'envelliments d'una variable}
\addfield{Actors:}{Usuari registrat}
\addscenario{Curs t\'{i}pic d'esdeveniments:}{
    \item L'usuari selecciona un conjunt de fitxers d'una variable d'un llistat.
    \item El sistema li mostra un vista de confirmaci\'{o} de la acci\'{o}.
    \item L'usuari confirma la acci\'{o}.
    \item El sistema esborra tots els fitxers.
}
\addscenario{Cursos alternatius:}{
\item[3] L'usuari cancela la acci\'{o}.
}
\end{usecase}

\subsubsection{Afegir un nou fitxer a un conjunt de fitxers d'envelliments}
\begin{usecase}
\addtitle{Variable006}{Afegir un nou fitxer a un conjunt de fitxers d'envelliments}
\addfield{Actors:}{Usuari registrat}
\addscenario{Curs t\'{i}pic d'esdeveniments:}{
    \item L'usuari selecciona una variable d'un llistat de variables.
    \item El sistema li mostra un formulari de edici\'{o} dels components.
    \item L'usuari pot donar un nom i seleccionar 0, 1 o 2 fitxers, on cada fitxer pot ser configurat com:
	\begin{itemize}
		\item a unic per tot l'any
		\item com estiu
		\item com hivern
		\item com tardor
		\item com estiu
	\end{itemize}
	\item L'usuari salva els canvis.
	\item El sistema guarda els canvis.
}
\end{usecase}

\subsubsection{Esborrar un fitxer d'un conjunt de fitxers d'envelliments}
\begin{usecase}
\addtitle{Variable007}{Esborrar un fitxer d'un conjunt de fitxers d'envelliments}
\addfield{Actors:}{Usuari registrat}
\addscenario{Curs t\'{i}pic d'esdeveniments:}{
    \item L'usuari selecciona una variable d'un llistat de variables.
    \item El sistema li mostra un formulari de edici\'{o} dels components.
	\item L'usuari selecciona un fitxer per esborrar.
	\item El sistema li demana confirmaci\'{o}.
	\item L'usuari confirma l'acci\'{o}
	\item El sistema esborra el fitxer.
}
\addscenario{Cursos alternatius:}{
\item[5] L'usuari cancela la acci\'{o}.
}
\end{usecase}


\subsubsection{Crear una matriu des d'un fitxer}
\begin{usecase}
\addtitle{Matriu001}{Crear una matriu des d'un fitxer}
\addfield{Actors:}{Usuari registrat}
\addscenario{Curs t\'{i}pic d'esdeveniments:}{
	\item L'usuari accedeix a un formulari on pot donar nom a la matriu i seleccionar el fitxer en format csv o excel. L'usuari accepta el formulari.
	\item El sistema crear la matriu on:
	\begin{itemize}
	\item La primera fila del fitxer s'associa com una variable Ichnaea. Si la variable es igual al identificador de la variable, automaticament s'assigna a aquesta columna a aquesta variable i a un conjunt de fitxers d'envelliments per defecte.
	\item Cada fila del fitxer, despr\'{e}s de la primera fila:
	\begin{itemize}
		\item La primera columna \'{e}s el identificador de la mostra. Si cont\'{e} un alias d'origen, autom\`{a}ticament s'assigna un origen
		\item Les columnes restants s'assignan com a valors de la mostra
	\end{itemize}
	\item L'usuari visualitza la matriu.
	\end{itemize}
}
\end{usecase}

\subsubsection{Actualitzar una matriu}
\begin{usecase}
\addtitle{Matriu002}{Actualitzar una matriu}
\addfield{Actors:}{Propietari de la matriu}
\addscenario{Curs t\'{i}pic d'esdeveniments:}{
	\item L'usuari selecciona una matriu per actualitzar.
	\item El sistema li mostra un formulari d'edici\'{o} de la matriu.
	\item L'usuari accedeix a un formulari on pot actualitzar el nom a la matriu i/o seleccionar el fitxer de tipus de fulla de c\`{a}lcul. L'usuari confirma els canvis
	\item El sistema esborra la matriu anterior i torna a crear la matriu on:
	\begin{itemize}
	\item La primera fila del fitxer s'associa com una variable Ichnaea. Si la variable es igual al identificador de la variable, automaticament s'assigna a aquesta columna a aquesta variable i a un conjunt de fitxers d'envelliments per defecte.
	\item Cada fila del fitxer, despr\'{e}s de la primera fila:
	\begin{itemize}
		\item La primera columna \'{e}s el identificador de la mostra. Si cont\'{e} un alias d'origen, autom\`{a}ticament s'assigna un origen
		\item Les columnes restants s'assignan com a valors de la mostra
	\end{itemize}
	\end{itemize}
	\item L'usuari visualitza la matriu.
}
\end{usecase}

\subsubsection{Clonar una matriu}
\begin{usecase}
\addtitle{Matriu003}{Clonar una matriu}
\addfield{Actors:}{Usuari registrat}
\addscenario{Curs t\'{i}pic d'esdeveniments:}{
	\item L'usuari selecciona una matriu per clonar.
	\item El sistema renderitza un formulari amb un nom suggerit per la matriu.
	\item L'usuari pot canviar el nom i acceptar la clonaci\'{o}.
	\item El sistema clona la matriu i la seva configuraci\'{o} sense copiar entrenaments ni prediccions. El propietari de la matriu \'{e}s l'usuari que ha realitzat la clonaci\'{o}. 
	\item L'usuari veu la matriu clonada.
}
\end{usecase}

\subsubsection{Esborrar una matriu}
\begin{usecase}
\addtitle{Matriu004}{Esborrar una matriu}
\addfield{Actors:}{Propietari de la matriu}
\addscenario{Curs t\'{i}pic d'esdeveniments:}{
	\item L'usuari selecciona una matriu del sistema per esborrara.
	\item El sistema demana confirmaci\'{o} per esborrar la matriu.
	\item L'usuari confirma la acci\'{o}.
	\item El sistema clona la matriu i la seva configuraci\'{o} sense copiar trainigs ni prediccions. El propietari de la matriu \'{e}s l'usuari que ha realitzat la clonaci\'{o}.
}
\addscenario{Cursos alternatius:}{
	\item[3] L'usuari cancela la acci\'{o}.
}
\end{usecase}

\subsubsection{Configurar una columna d'una matriu}
\begin{usecase}
\addtitle{Matriu005}{Configurar la columna d'una matriu}
\addfield{Actors:}{Usuari propietari de la matriu}
\addscenario{Curs t\'{i}pic d'esdeveniments:}{
	\item L'usuari selecciona una matriu
	\item El sistema mostra una vista per configurar les columnes de una matriu.
	\item L'usuari selecciona una columna i pot:
	\begin{itemize}
		\item donar un nom a la columna
		\item seleccionar una variable i una conjunt d'envelliments de la variable
	\end{itemize}
	\item L'usuari accepta la configuraci\'{o}
	\item El sistema salva la configuraci\'{o} de la columna
}
\end{usecase}


\subsubsection{Configurar una mostra d'una matriu}
\begin{usecase}
\addtitle{Matriu 004}{Configurar una mostra d'una matriu}
\addfield{Actors:}{Usuari propietari de una matriu}
\addscenario{Curs t\'{i}pic d'esdeveniments:}{
	\item L'usuari selecciona un matriu.
	\item El sistema mostra una vista per configurar les mostres de la matriu.
	\item L'usuari selecciona una mostra i pot:
	\begin{itemize}
	\item donar una data 
	\item donar una nom a la mostra
	\item donar un origen de la mostra
	\end{itemize}
	\item L'usuari accepta la configuraci\'{o}.
	\item El sistema guarda la configuraci\'{o} de la mostra.
}
\end{usecase}

\subsubsection{Llistar els entrenaments del sistema}
\begin{usecase}
\addtitle{Training001}{Llistar els entrenaments del sistema}
\addfield{Actors:}{Usuari administrador}
\addscenario{Curs t\'{i}pic d'esdeveniments:}{
	\item L'usuari accedeix a la vista del llistat de entrenaments del sistema.
	\item El sistema llista els trainings amb les dades b\`{a}siques: matriu entrenada, estat del training, descripci\'{o} del training, creador, data de creaci\'{o}.
}
\end{usecase}

\subsubsection{Llistar els meus entrenaments}
\begin{usecase}
\addtitle{Training002}{Llistar els meus entrenaments}
\addfield{Actors:}{Usuari registrat}
\addscenario{Curs t\'{i}pic d'esdeveniments:}{
	\item L'usuari accedeix a la vista del llistat de entrenaments que ha creat.
	\item El sistema llista els trainings que ha creat amb dades b\`{a}siques: matriu entrenada, estat del training, descripci\'{o} del training, data de creaci\'{o}.
}
\end{usecase}

\subsubsection{Llistar matrius entrenables}
\begin{usecase}
\addtitle{Training003}{Llistar matrius entrenables}
\addfield{Actors:}{Usuari registrat}
\addscenario{Curs t\'{i}pic d'esdeveniments:}{
	\item L'usuari accedeix a la vista del llistat de matrius entrenables.
	\item El sistema llista els entrenaments amb un estat finalitzat i sense errors amb dades b\`{a}siques: matriu entrenada, estat del training, creador, descripci\'{o} del training, data de creaci\'{o}.
}
\end{usecase}

\subsubsection{Crear un entrenament}
\begin{usecase}
\addtitle{Training004}{Crear un entrenament}
\addfield{Actors:}{Usuari registrat}
\addscenario{Curs t\'{i}pic d'esdeveniments:}{
	\item L'usuari selecciona una matriu per entrenar
	\item El sistema li mostra un formulari per crear entrenaments.
	\item L'usuari pot donar un nom, una descripci\'{o}, seleccionar un origen dels disponibles i quines columnes vol entrenar. Finalment confirma les dades.
	\item El sistema guarda el training i envia al sistema de cues les dades. El sistema evalua si ha pogut enviar el training al sistema de cues en cas que el servei estigui caigut. 
	\item L'usuari veu el training creat.
}
\end{usecase}

\subsubsection{Reenviar un entrenament al sistema de cues}
\begin{usecase}
\addtitle{Training005}{Reenviar un entrenament al sistema de cues}
\addfield{Actors:}{Usuari propietari d'un entrenament}
\addscenario{Curs t\'{i}pic d'esdeveniments:}{
    \item L'usuari selecciona un entrenament que ha tingut problemes de enviament.
    \item El sistema renderitza una vista de visualitzaci\'{o} del entrenament.
    \item L'usuari pot consultar quin possible error ha passat i pot confirmar el reenviament.
    \item El sistema actulitza les dades i reenvia les dades al sistema de cues.
}
\end{usecase}

\subsubsection{Visualitzar un entrenament}
\begin{usecase}
\addtitle{Training006}{Visualitzar un entrenament}
\addfield{Actors:}{Usuari registrat}
\addscenario{Curs t\'{i}pic d'esdeveniments:}{
    \item L'usuari selecciona d'un llistat un entrenament.
    \item El sistema renderitza una vista de visualitzaci\'{o} del training amb el nom, descripci\'{o}, data de creaci\'{o} i  errors o resultats segons el cas.
}
\end{usecase}

\subsubsection{Esborrar un entrenament}
\begin{usecase}
\addtitle{Training007}{Esborrar un entrenament}
\addfield{Actors:}{Usuari superadministrador, usuari propietari d'un entrenament}
\addscenario{Curs t\'{i}pic d'esdeveniments:}{
	\item L'usuari selecciona un entrenament per esborrar.
	\item El sistema demana confirmaci\'{o} per esborrar el training.
	\item L'usuari confirma l'acci\'{o}
	\item El sistema esborra el entrenaments i totes les prediccions que s'han fet a partir d'aquest entrenament.
}
\end{usecase}

\subsubsection{Descarregar els resultats d'un entrenament}
\begin{usecase}
\addtitle{Training008}{Descarregar els resultats d'un entrenament}
\addfield{Actors:}{Usuari registrat}
\addscenario{Curs t\'{i}pic d'esdeveniments:}{
    \item L'usuari selecciona un training finalitzat
    \item El sistema visualitza un enlla� amb la possibilitat de descarregar el fitxer resultats d'un entrenament.
    \item L'usuari accedeix a la descarrega.
    \item El sistema envia a l'usuari els resultats.
}
\end{usecase}

\subsubsection{Actualitzar l'estat d'un entrenament}
\begin{usecase}
\addtitle{Training009}{Actualitzar l'estat d'un entrenament}
\addfield{Actors:}{Cua}
\addscenario{Curs t\'{i}pic d'esdeveniments:}{
    \item La cua avisa al consumidor que ha finalitzat un entrenament i li envia unes dades al consumidor
    \item El consumidor rep les dades i li envia al sistema
    \item El sistema les guarda i actulitza l'estat del entrenament.
}
\end{usecase}

\subsubsection{Llistar prediccions del sistema}
\begin{usecase}
\addtitle{Prediction001}{Llistar prediccions del sistema}
\addfield{Actors:}{Usuari administrador}
\addscenario{Curs t\'{i}pic d'esdeveniments:}{
	\item L'usuari accedeix a la vista del llistat de prediccions
	\item El sistema llista totes les prediccions amb dades b\`{a}siques.
}
\end{usecase}

\subsubsection{Llistar les meves prediccions}
\begin{usecase}
\addtitle{Prediction002}{Llistar les meves prediccions}
\addfield{Actors:}{Usuari registrats}
\addscenario{Curs t\'{i}pic d'esdeveniments:}{
	\item L'usuari accedeix a la vista del llistat de prediccions creades per ell.
	\item EL sistema llista totes les prediccions creades per l'usuari amb dades b\`{a}siques.
}
\end{usecase}

\subsubsection{Esborrar una predicci\'{o}}
\begin{usecase}
\addtitle{Prediction003}{Esborrar una predicci\'{o}}
\addfield{Actors:}{Usuari registrats}
\addscenario{Curs t\'{i}pic d'esdeveniments:}{
	\item L'usuari selecciona una predicci\'{o} per esborrar.
	\item El sistema li demana confirmaci\'{o} per esborrar la predicci\'{o}.
	\item L'usuari confirma la acci\'{o}.
	\item El sistema esborra la predicci\'{o}.
}
\addscenario{Cursos alternatius:}{
	\item[3] L'usuari cancela la acci\'{o}.
}
\end{usecase}

\subsubsection{Crear una matriu de predicci\'{o} des d'un fitxer}
\begin{usecase}
\addtitle{Prediction004}{Crear una matriu de predicci\'{o} des d'un fitxer}
\addfield{Actors:}{Usuari registrats}
\addscenario{Curs t\'{i}pic d'esdeveniments:}{
	\item L'usuari selecciona un entrenament per crear una predicci\'{o}.
	\item El sistema li mostra un formulari de creaci\'{o} de prediccions.
	\item L'usuari accedeix a un formulari on pot donar nom a la matriu de predicci\'{o} i seleccionar una fulla de c\`{a}lcul. L'usuari accepta el formulari.
	\item El sistema crear la matriu on:
	\begin{itemize}
	\item Cada fila del fitxer, despr\'{e}s de la primera fila:
	\begin{itemize}
		\item La primera columna \'{e}s el identificador de la mostra. Si cont\'{e} un alias d'origen, autom\`{a}ticament s'assigna un origen
		\item Les columnes restants s'assignan com a valors de la mostra
	\end{itemize}
	\item L'usuari visualitza la matriu.
	\end{itemize}
}
\end{usecase}

\subsubsection{Actualitzar una matriu de predicci\'{o} des d'un fitxer}
\begin{usecase}
\addtitle{Prediction005}{Actualitzar una matriu de predicci\'{o} desde un fitxer}
\addfield{Actors:}{Usuari propietari de la predicci\'{o}}
\addscenario{Curs t\'{i}pic d'esdeveniments:}{
    \item L'usuari seleccion un entrenament
    \item El sistema li mostra un formulari d'edici\'{o} de prediccions.
    \item L'usuari pot actualitzar el nom de la predicci\'{o} i seleccionar una fulla de c\'{a}lcul i guardar els canvis.
    \item El sistema esborra la matrius de predicci\'{o}, actualitza les dades i crea la matriu on:
    	\begin{itemize}
	\item Cada fila de la fulla de calcul, despr\'{e}s de la primera fila:
	\begin{itemize}
		\item La primera columna \'{e}s el identificador de la mostra. Si cont\'{e} un alias d'origen, autom\`{a}ticament s'assigna un origen
		\item Les columnes restants s'assignan com a valors de la mostra
	\end{itemize}
	\item L'usuari visualitza la matriu.
	\end{itemize}
}
\end{usecase}

\subsubsection{Configurar una mostra d'una matriu de predicci\'{o}}
\begin{usecase}
\addtitle{Prediction 006}{Configurar una mostra d'una matriu de predicci\'{o}}
\addfield{Actors:}{Usuari propietari de la predicci\'{o}}
\addscenario{Curs t\'{i}pic d'esdeveniments:}{
	\item L'usuari selecciona una predicci\'{o} i una mostra.
	\item El sistema mostra una vista de edici\'{o} de la matriu.
	\item L'usuari pot donar un data, un nom i un origen a la mostra i guardar els canvis.
	\item El sistema guarda les dades.
}
\end{usecase}

\subsubsection{Enviar una predicci\'{o} al sistema de cues}
\begin{usecase}
\addtitle{Prediction007}{Enviar una predicci\'{o} al sistema de cues}
\addfield{Actors:}{Usuari propietari de la predicci\'{o}}
\addscenario{Curs t\'{i}pic d'esdeveniments:}{
    \item L'usuari selecciona una predicci\'{o}.
    \item El sistema li mostra un formulari per enviar la matriu de predicci\'{o} al sistema de cues.
    \item L'usuari confirma la acci\'{o}.
    \item El sitema prepara les dades per enviar i li dona al sistema de cues.
}
\end{usecase}

\subsubsection{Veure una predicci\'{o}}
\begin{usecase}
\addtitle{Prediction008}{Veure una predicci\'{o}}
\addfield{Actors:}{Usuari registrat}
\addscenario{Curs t\'{i}pic d'esdeveniments:}{
	\item L'usuari selecciona una predicci\'{o}.
	\item El sistema li mostra les dades b\`{a}siques de la predicci\'{o} i la matriu de predicci\'{o}.
}
\end{usecase}

\subsubsection{Veure els resultats d'una predicci\'{o}}
\begin{usecase}
\addtitle{Prediction009}{Veure els resultats d'una predicci\'{o}}
\addfield{Actors:}{Usuari registrat}
\addscenario{Curs t\'{i}pic d'esdeveniments:}{
	\item L'usuari selecciona els resultats d'una predicci\'{o}.
	\item El sistema li mostra les diferents sortides de la execuci\'{o}.
}
\end{usecase}

\subsubsection{Actualitzar l'estat d'una predicci\'{o}}
\begin{usecase}
\addtitle{Prediction010}{Actualitzar l'estat d'una predicci\'{o}}
\addfield{Actors:}{Cua, Consumidor}
\addscenario{Curs t\'{i}pic d'esdeveniments:}{
    \item La cua avisa al consumidor que ha finalitzat una predicci\'{o} i li envia les dades al consumidor
    \item El consumidor rep les dades i li envia al sistema
    \item El sistema les guarda i actulitza l'estat de la predicci\'{o}.
}
\end{usecase}

\section{Model Conceptual}
A continuaci\'{o} es descriur\`{a} el model 

\subsection{Diagrama de classes}
La figura \ref{fig:datamodel} \'{e}s el diagrama UML on est\`{a} implementat el modelatge de dades.
\begin{sidewaysfigure}
  \includegraphics[scale=0.5]{img/specification/ModelClass.png}
  \caption{Model de dades}
  \label{fig:datamodel}
\end{sidewaysfigure}


\subsection{Explicaci\'{o} de les classes}
A continuaci\'{o} es descriuran algunes de le classes i alguns atributs per a la seva millor comprensi\'{o}.
\begin{itemize}
\item \textit{Users}: en aquesta classe es guarda la informaci\'{o} dels usuaris.

\item \textit{Groups}: en aquesta classe es guarda la informaci\'{o} dels grups. 

\item \textit{Season}: en aquesta classe es guarda la informaci\'{o} d'un fitxer d'envelliment.

\item \textit{SeasonSet}: en aquesta classe es guarda la informaci\'{o} d'un conjunt de fitxers.

\item \textit{SeasonSetComponent}: aquesta classe associa fitxers als conjunts de fitxers.

\item \textit{Matrix}: en aquesta classe es guarda la informaci\'{o} b\`{a}sica d'una matriu.

\item \textit{Sample}:en aquesta classe es guarda la informaci\'{o} de cadascuna de les files de la matriu a la que est\'{a} asociada mitjan�ant la associaci\'{o} \textit{rows}.

\item \textit{VariableMatrixConfig}: aquesta classe guarda la configuraci\'{o} d'una columna de la matriu i associa la columna a una variable i a un conjunt de fitxers d'aquesta variable.

\item \textit{Training}: en aquesta classe es guarda la informaci\'{o} d'un entrenament.
\begin{itemize}
\item requestId: identificador del proc\'{e}s en la cua d'execuci\'{o}.
\item status: estat de la predicci\'{o}. Mirar l'apartat \ref{sec:status}.
\end{itemize}

\item \textit{ColumnTrained}: aquesta classe associa un entrenament amb quines columnes es volen entrenar.

\item \textit{PredictionMatrix}: en aquesta classe es guarda la informaci\'{o} d'una matriu de predicci\'{o}.
\begin{itemize}
\item requestId: identificador del proc\'{e}s en la cua d'execuci\'{o}.
\item predictionResult: collecci\'{o} de resultats de la execuci\'{o} d'una predicci\'{o}.
\item status: estat de la predicci\'{o}. Mirar \ref{specificaion:status}.
\end{itemize}

\item \textit{PredictionSample}: en aquesta classe es guarda la informaci\'{o} de les mostres(files) d'una matriu de predicci\'{o}.

\end{itemize}

\section{Model d'estats}
\label{sec:status}
La aplicaci\'{o} i el sistema de cues son dos components separats. Aix\'{o} ens obliga a definir estats per els entrenaments i per les prediccions. El principal problema que existeix en la versi\'{o} actual de Ichnaea Software es que no retorna estats parcials.

\subsection{Estats dels entrenaments}
\begin{figure}[H]
  \centering
  \includegraphics[scale=0.4]{img/specification/StatesTraining.png}
  \caption{Diagram d'estats dels entrenaments}
  \label{fig:statestraining}
\end{figure}
Per crear un entrenament, primer es comproba que es pot establir connexi\'{o} amb la cua. Si no \'{e}s pot, l'entrenament queda marcat amb l'estat pendent(\textit{pending}). En el cas que es pugui establir connexi\'{o}, s'envia les dades i es queda en aquest fins que li arribi algun event de finalitzaci\'{o} amb els resultats o amb un error d'Ichnaea.\\
En el cas que estigui pendent, l'usuari podr\'{a} reintentar enviar l'entrenament a la cua despr\'{e}s de comprovar l'error.

\subsection{Estats de les prediccions}
\begin{figure}[H]
  \centering
  \includegraphics[scale=0.4]{img/specification/StatesPrediction.png}
  \caption{Diagram d'estats dels entrenaments}
  \label{fig:statestraining}
\end{figure}
Per crear una predicci\'{o}, primer s'ha de configurar la matriu. Mentre l'usuari estigui configurant la predicci\'{o} estar\'{a} en un estat inicial(\textit{new}). Quan l'usuari confirmi l'enviament, primer es comproba que es pot establir connexi\'{o} amb la cua. Si no \'{e}s pot, l'entrenament queda marcat amb l'estat pendent(\textit{pending}). En el cas que es pugui establir connexi\'{o}, s'envia les dades. Es queda en aquest estat rebent multiples respostes fins que li arribi algun event de finalitzaci\'{o} amb els resultats o amb els errors.\\
En el cas que estigui pendent, l'usuari podr\'{a} reintentar enviar la predicci\'{o} a la cua despr\'{e}s de comprovar l'error.

\section{Model del comportament}
En aquesta secci\'{o} donarem veurem els diagrames de seq\�{u}encia de les operacions m\'{e}s complexes dels casos d'us.
\subsection*{Crear un conjunt de fitxers d'envelliments per una variable}
\begin{figure}[H]
  \centering
  \includegraphics[scale=0.6]{img/specification/SequenceNewSeasonSe.png}
  \caption{Diagram de seq�{u}encia \textit{Crear un conjunt de fitxers d'envelliments per una variable}}
  \label{fig:sequencenewseasonset}
\end{figure}

\subsection*{Afegir un nou fitxer a un conjunt de fitxers d'envelliments}
\begin{figure}[H]
  \centering
  \includegraphics[scale=0.6]{img/specification/SequenceAddNewFilesToSeasonSet.png}
  \caption{Diagram de seq�{u}encia \textit{Afegir un nou fitxer a un conjunt de fitxers d'envelliments}}
  \label{fig:sequenceaddnewseason}
\end{figure}

\subsection*{Crear una matriu des d'un fitxer}
\begin{figure}[H]
  \centering
  \includegraphics[scale=0.6]{img/specification/SequenceCreateMatrix.png}
  \caption{Diagram de seq�{u}encia \textit{Crear una matriu des d'un fitxer}}
  \label{fig:sequencenewmatrixfromfile}
\end{figure}

\subsection*{Crear un entrenament}
\begin{figure}[H]
  \centering
  \includegraphics[scale=0.6]{img/specification/SequenceCreateTraining.png}
  \caption{Diagram de seq�{u}encia \textit{Crear un entrenament}}
  \label{fig:sequenceaddtraining}
\end{figure}

\subsection*{Configurar una columna d'una matriu}
\begin{figure}[H]
  \centering
  \includegraphics[scale=0.6]{img/specification/SequenceConfigureColumn.png}
  \caption{Diagrama de seq�{u}encia \textit{Configurar una columna d'una matriu} }
  \label{fig:sequenceconfigurecolumn}
\end{figure}

\subsection*{Actualitzar l'estat d'una predicci\'{o}}