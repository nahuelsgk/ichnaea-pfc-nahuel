\chapter{Especificaci\'{o}}
\label{cha:specification}
En aquest cap\'{i}tol es descriuen els requeriments, les operatives que es necessiten i el model de dades.

\section{Requeriments funcionals}

\subsection{Administraci\'{o} d'usuaris}
La aplicaci\'{o} ha de estar protegida i autoritzada pels usuaris. Els usuaris autenticats han de tenir permisos i  pertanyer a grups amb rols autoritzats per fer certes accions. \'{E}s una aplicaci\'{o} distribuida per tant s'ha de dissenyar un sistema que permeti:
\begin{itemize}
\item Crear comptes d'usuari.
\item Atendre peticions de resetejar contrasenyes.
\item Enviar mails de confirmacions de accions.
\item Canviar permisos a usuaris.
\end{itemize}

\subsection{Administraci\'{o} de variables de sistema}
La aplicaci\'{o} ha de gestionar les variables que Ichnaea al sistema per poder utilitzar-les amb el software. Les variables han de tenir asociades un o varis conjunt ''seasons''\ref{cha:backgroud:univers:matrius:variables_seasons}. La aplicaci\'{o} ha de poder gestionar les variables, els continguts de les ''seasons'' i les associacions entre les variables i els conjunts de ''season''.

\subsection{Administraci\'{o} de matrius}
La aplicaci\'{o} ha de poder gestionar i configurar matrius de dades. Per la creaci\'{o} de matrius ha de poder llegir un fitxer csv o excel i crear un model de dades que representi una matriu a partir de les dades proporcionades.\\

La aplicaci\'{o} permet la configuraci\'{o} de matrius\ref{cha:backgroud:univers:matrius}:
\begin{itemize}
\item Configurar les columnes com una variable i una ''season set''.
\item Configurar l'origen d'un ''sample''.
\item Configurar la data d'un ''sample''.
\item Configurar o actualitzar dades.
\end{itemize}

\subsection{Administraci\'{o} de trainings}
La aplicaci\'{o} ha de:
\begin{itemize}
	\item Gestionar ''trainings''.
	\item Enviar a processar-los a la cua de processos.
	\item Llegir e interpretar el estat del proc\'{e}s.
	\item Llegir els resultats dels ''trainings''.
\end{itemize}

\subsection{Administraci\'{o} de matrius de predicci\'{o}}
La aplicaci\'{o} ha de:
\begin{itemize}
	\item Gestionar les matrius de prediccions 
	\item Enviar a processar-los a la cua de processos
	\item Llegir e interpretar el estat del proc\'{e}s 
	\item Llegir els resultats de les prediccions
\end{itemize}

\section{Requeriments no funcionals}
Els requeriments no funcionals son:
\begin{itemize}
\item Bon rendiment
\item Escalabilitat
\item Mantenible
\item Flexibilitat
\end{itemize}

\section{Casos d'us i fluxos}
En la documentaci\'{o} s'utilitzar\`{a} la seg\�{u}ent estructura per definir els casos d'\'{u}s:\\

\begin{usecase}
\addtitle{Identificador}{Nom cas d'us}
\addfield{Actors:}{Llista de actors}
\addscenario{Curs tipic d'esdeveniments:}{
	\item Esdeveniment
	\item Esdeveniment
	\item ...
}
\addscenario{Cursos alternatius:}{
\item Esdeveniment Alternatiu
}
\end{usecase}


\subsection{Crear un usuari}
\begin{usecase}
\addtitle{Usuari001}{Crear un usuari}
\addfield{Actors:}{Anonim}
\addscenario{Curs tipic d'esdeveniments:}{
	\item Usuari accedeix al formulari de registraci\'{o}. L'usuari introdueix un nom d'usuari, un correu electr\`{o}nic i una contrasenya per duplicat
	\item El sistema envia al usuari una confirmaci\'{o} via correu electr\'{o}nic amb un enlla� de confirmaci\'{o} i crea un compte no validada.
	\item L'usuari accedeix mitjan�ant l'enlla� de confirmaci\'{o}.
	\item El sistema comprova que \'{e}s un enlla� de confirmaci\'{o} v\`{a}lid d'aquest usuari i activa la compte. L'usuari ja est\'{a} autenticat al sistema com un usuari i ja \'{e}s un usuari del sistema.
}
\addscenario{Cursos alternatius:}{
\item[3] El sistema valida que no existeixi un usuari amb aquesta compte de correu, que el correu sigui v\`{a}lid. Sino es correcte li informa a l'usuari al mateix formulari.
}
\end{usecase}

\subsection{Canviar un usuari de grup}
\begin{usecase}
\addtitle{Usuari002}{Canviar un usuari de grup}
\addfield{Actors:}{Anonim}
\addscenario{Curs tipic d'esdeveniments:}{
	\item L'administrador llista tots els usuaris del sistema i selecciona un.
	\item L'administrador veu el formulari de edici\'{o} de permisos.
	\item L'administrador selecciona el nou perm\'{s} i salva el perfil.
	\item El sistema guarda el nou perm\'{i}s
}
\end{usecase}

\subsection{Crear una variable}
\begin{usecase}
\addtitle{Variable001}{Crear una variable}
\addfield{Actors:}{Usuari administrador}
\addscenario{Curs tipic d'esdeveniments:}{
	\item L'usuari visualitza el formulari on pot donar un identificador que ha de ser \'{u}nic en el sistema i una descripci\'{o} .
	\item El sistema crear la variable.
}
\end{usecase}

\subsection{Actualitzar una variable}
\begin{usecase}
\addtitle{Variable002}{Actualitzar una variable}
\addfield{Actors:}{Usuari administrador}
\addscenario{Curs tipic d'esdeveniments:}{
    \item L'admininistrador seleccion un variable d'un llistat de variables.
    \item El sistema li mostra un formulari d'edici\'{o} on pot veure, les ''season sets'' i els seus components(''seasons'') i pot actualitzar la descripci\'{o}.
    \item L'usuari modifica la descripci\'{o} i salva els canvis.
    \item El sistema guarda les modificacions.
}
\end{usecase}

\subsection{Crear una season set}
\begin{usecase}
\addtitle{Variable003}{Crear una ''season set''}
\addfield{Actors:}{Usuari registrat}
\addscenario{Curs tipic d'esdeveniments:}{
	\item L'usuari selecciona una variable d'un llistat de variables.
	\item El sistema li mostra un formulari d'edici\'{o} de la variable amb un enlla� a un formulari de creaci\'{o}.
	\item L'usuari accedeix a un formulari de creaci\'{o} de ''season set''.
	\item El sistema li mostra un formulari de creaci\'{o}.
	\item L'usuari pot donar un nom i seleccionar 0, 1 o 2 fitxers amb el contingut dels ''seasons'', on cada fitxer pot ser configurat com:
	\begin{itemize}
		\item a unic per tot l'any
		\item com estiu
		\item com hivern
		\item com tardor
		\item com estiu
	\end{itemize}	
	\item L'usuari salva els canvis.
	\item El sistema guarda els canvis.
}
\end{usecase}

\subsection{Actualitzar una ''season set''}
\begin{usecase}
\addtitle{Variable004}{Actualitzar una ''season set''}
\addfield{Actors:}{Usuari registrat}
\addscenario{Curs tipic d'esdeveniments:}{
    \item L'usuari selecciona una variable d'un llistat de variables.
    \item El sistema li mostra un formulari d'edici\'{o} on pot editar el nom.
    \item L'usuari selecciona una ''season set''.
    \item El sistema li mostra un formulari on pot actualitzar una ''season set''
    \item L'usuari actualiza les dades i guarda els canvis.
    \item El sistema guarda els canvis. 
}
\end{usecase}

\subsection{Esborrar una ''season set''}
\begin{usecase}
\addtitle{Variable005}{Crear una ''season set''}
\addfield{Actors:}{Usuari registrat}
\addscenario{Curs tipic d'esdeveniments:}{
    \item L'usuari selecciona una variable d'un llistat de variables per esborrar.
    \item El sistema li mostra un vista de confirmaci\'{o} de la acci\'{o}.
    \item L'usuari confirma la acci\'{o}.
    \item El sistema esborra una ''season set'' i els seus components(''seasons'').
}
\end{usecase}

\subsection{Afegir una nova ''season'' a una ''season set''}
\begin{usecase}
\addtitle{Variable006}{Afegir una nova ''season'' d'una ''season set''}
\addfield{Actors:}{Usuari registrat}
\addscenario{Curs tipic d'esdeveniments:}{
    \item L'usuari selecciona una variable d'un llistat de variables.
    \item El sistema li mostra un formulari de edici\'{o} dels components(''seasons'').
    \item L'usuari pot donar un nom i seleccionar 0, 1 o 2 fitxers amb el contingut dels ''seasons'', on cada fitxer pot ser configurat com:
	\begin{itemize}
		\item a unic per tot l'any
		\item com estiu
		\item com hivern
		\item com tardor
		\item com estiu
	\end{itemize}
	\item L'usuari salva els canvis.
	\item El sistema guarda els canvis.
}
\end{usecase}

\subsection{Esborrar una ''season'' d'una ''season set''}
\begin{usecase}
\addtitle{Variable007}{Esborrar una ''season'' d'una ''season set''}
\addfield{Actors:}{Usuari registrat}
\addscenario{Curs tipic d'esdeveniments:}{
    \item L'usuari selecciona una variable d'un llistat de variables.
    \item El sistema li mostra un formulari de edici\'{o} dels components(''seasons'').
	\item L'usuari selecciona una ''season'' per esborrar.
	\item El sistema li demana confirmaci\'{o}.
	\item L'usuari confirma l'acci\'{o}
	\item El sistema esborra el component ''season''
}
\end{usecase}


\subsection{Crear un matriu des d'un fitxer}
\begin{usecase}
\addtitle{Matriu 001}{Crear una matrius des d'un fitxer}
\addfield{Actors:}{Usuari registrat}
\addscenario{Curs tipic d'esdeveniments:}{
	\item L'usuari accedeix a un formulari on pot donar nom a la matriu i seleccionar el fitxer en format csv o excel. L'usuari accepta el formulari.
	\item El sistema crear la matriu on:
	\begin{itemize}
	\item La primera fila del fitxer s'associa com una variable Ichnaea. Si la variable es igual al identificador de la variable, automaticament s'assigna a aquesta columna a aquesta variable i a un conjunt de ''seasons''(''season sets'') per defecte.
	\item Cada fila del fitxer, despr\'{e}s de la primera fila:
	\begin{itemize}
		\item La primera columna \'{e}s el identificador de la mostra. Si cont\'{e} un alias d'origen, autom\`{a}ticament s'assigna un origen
		\item Les columnes restants s'assignan com a valors de la mostra
	\end{itemize}
	\item L'usuari visualitza la matriu.
	\end{itemize}
}
\end{usecase}

\subsection{Actualitzar una matriu}
\begin{usecase}
\addtitle{Matriu 002}{Actualitzar una matriu}
\addfield{Actors:}{Propietari de la matriu}
\addscenario{Curs tipic d'esdeveniments:}{
	\item L'usuari accedeix a un formulari on pot actualitzar el nom a la matriu i/o seleccionar el fitxer de tipus de fulla de c\`{a}lcul. L'usuari confirma els canvis
	\item El sistema esborra la matriu anterior i torna a crear la matriu seguint el cas d'us ''Matriu001''.
}
\end{usecase}

\subsection{Clonar una matriu}
\begin{usecase}
\addtitle{Matriu003}{Clonar una matriu}
\addfield{Actors:}{Usuari registrat}
\addscenario{Curs tipic d'esdeveniments:}{
	\item L'usuari selecciona una matriu per clonar.
	\item El sistema renderitza un formulari amb un nom suggerit per la matriu.
	\item L'usuari pot canviar el nom i acceptar la clonaci\'{o}.
	\item El sistema clona la matriu i la seva configuraci\'{o} sense copiar ''trainings'' ni prediccions. El propietari de la matriu \'{e}s l'usuari que ha realitzat la clonaci\'{o}. 
	\item L'usuari veu la matriu clonada.
}
\end{usecase}

\subsection{Esborra una matriu}
\begin{usecase}
\addtitle{Matriu004}{Esborrar una matriu}
\addfield{Actors:}{Propietari de la matriu}
\addscenario{Curs tipic d'esdeveniments:}{
	\item L'usuari selecciona una matriu del sistema per esborrara.
	\item El sistema demana confirmaci\'{o} per esborrar la matriu.
	\item L'usuari confirma la acci\'{o}.
	\item El sistema clona la matriu i la seva configuraci\'{o} sense copiar trainigs ni prediccions. El propietari de la matriu \'{e}s l'usuari que ha realitzat la clonaci\'{o}.
}
\end{usecase}

\subsection{Configurar una columna de la matriu}
\begin{usecase}
\addtitle{Matriu 003}{Configurar la columna de una matriu}
\addfield{Actors:}{Usuari propietari de la matriu}
\addscenario{Curs tipic d'esdeveniments:}{
	\item L'usuari selecciona una matrius d'un llistat per configurar-la
	\item El sistema renderitza una vista per configurar les columnes de una matriu.
	\item L'usuari pot canviar d'una columna:
	\begin{itemize}
		\item un alias
		\item seleccionar una variable i una 'season set'
	\end{itemize}
	\item L'usuari accepta la configuraci\'{o}
	\item El sistema salva la configuraci\'{o} de la columna
}
\end{usecase}


\subsection{Configurar la data del ''sample'' d'una matriu}
\begin{usecase}
\addtitle{Matriu 004}{Configurar la data del ''sample'' de una matriu}
\addfield{Actors:}{Usuari propietari de una matriu}
\addscenario{Curs tipic d'esdeveniments:}{
	\item L'usuari selecciona un matriu i un ''sample''
	\item El sistema mostra un calendari per donar una data.
	\item L'usuari pot seleccionar una data o pot escriure la data. L'usuari confirma la data.
	\item El sistema guarda la data.
}
\end{usecase}

\subsection{Configurar l'origen de una ''sample''}
\begin{usecase}
\addtitle{Sample 001}{Configurar la data del ''sample'' de una matriu}
\addfield{Actors:}{Usuari propietari de una matriu}
\addscenario{Curs tipic d'esdeveniments:}{
	\item L'usuari selecciona un matriu i un ''sample''
	\item El sistema mostra un camp per poder escriure.
	\item L'usuari escriu un origen o selecciona un que li suggerireix el sistema. L'usuari confirma la dada.
	\item El sistema guarda la dada.
}
\end{usecase}

\subsection{Llistar ''trainings'' del sistema}
\begin{usecase}
\addtitle{Training 001}{Llistar ''trainings''}
\addfield{Actors:}{Usuari administrador}
\addscenario{Curs tipic d'esdeveniments:}{
	\item L'usuari accedeix a la vista del llistat de ''trainings''
	\item El sistema llista els trainings amb dades b\`{a}siques. Matriu entrenada, estat del training i descripci\'{o} del training.
}
\end{usecase}

\subsection{Llistar els meus ''trainings''}
\begin{usecase}
\addtitle{Training 002}{Llistar trainings}
\addfield{Actors:}{Usuari registrat}
\addscenario{Curs tipic d'esdeveniments:}{
	\item L'usuari accedeix a la vista del llistat de trainings
	\item El sistema llista els trainings amb dades b\`{a}siques. Matriu entrenada, estat del training i descripci\'{o} del training.
}
\end{usecase}

\subsection{Llistar ''trainings'' entrenables}
\begin{usecase}
\addtitle{Training 003}{Llistar ''trainings'' entrenables}
\addfield{Actors:}{Usuari registrat}
\addscenario{Curs tipic d'esdeveniments:}{
	\item L'usuari accedeix a la vista del llistat de trainings
	\item El sistema llista els trainings amb dades b\`{a}siques. Matriu entrenada, estat del training i descripci\'{o} del training.
}
\end{usecase}

\subsection{Crear un ''training''}
\begin{usecase}
\addtitle{Training 004}{Crear un training}
\addfield{Actors:}{Usuari registrat}
\addscenario{Curs tipic d'esdeveniments:}{
	\item L'usuari selecciona una matriu per entrenar
	\item El sistema li mostra un formulari per crear trainings
	\item L'usuari pot donar un nom, una descripci\'{o}, seleccionar un origen i quines columnes vol entrenar. Confirma les dades.
	\item El sistema guarda el training i envia al sistema de cues. El sistema evalua si ha pogut enviar el training al sistema de cues en cas que el servei estigui caigut. Renderitza la vista de visualitzaci\'{o} del training.
}
\end{usecase}

\subsection{Reenviar un training al sistema de cues}
\begin{usecase}
\addtitle{Training 005}{Reenviar un training}
\addfield{Actors:}{Usuari propietari d'un ''training''}
\addscenario{Curs tipic d'esdeveniments:}{
    \item L'usuari selecciona un training que ha tingut problemes de enviament.
    \item El sistema renderitza una vista de visualitzaci\'{o} del training.
    \item L'usuari pot consultar quin possible error ha passat i pot confirmar el reenviament
    \item El sistema actulitza les dades i reenvia les dades al sistema de cues.
}
\end{usecase}

\subsection{Visualitzar un training}
\begin{usecase}
\addtitle{Training 006}{Visualitzar un training}
\addfield{Actors:}{Usuari registrat}
\addscenario{Curs tipic d'esdeveniments:}{
    \item L'usuari selecciona d'un llistat un training.
    \item El sistema renderitza una vista de visualitzaci\'{o} del training amb el nom, descripci\'{o}, data de creaci\'{o} i  errors o resultats segons el cas.
}
\end{usecase}

\subsection{Esborrar un ''training''}
\begin{usecase}
\addtitle{Training 007}{Esborrar un ''training''}
\addfield{Actors:}{Usuari superadministrador, usuari propietari d'un ''training''}
\addscenario{Curs tipic d'esdeveniments:}{
}
\end{usecase}

\subsection{Descarregar els resultats d'un ''training''}
\begin{usecase}
\addtitle{Training 008}{Descarregar els resultats d'un ''training''}
\addfield{Actors:}{Usuari registrat}
\addscenario{Curs tipic d'esdeveniments:}{
    \item L'usuari selecciona un training finalitzat
    \item El sistema renderitza una vista de visualitzaci\'{o} del ''training''. Si el resultat es correcte, el sistema mostra una operaci\'{o} de descarrega dels resultats de un training.
    \item L'usuari accedeix a la descarrega.
    \item El sistema envia a l'usuari els resultats.
}
\end{usecase}

\subsection{Actualitzar l'estat d'un ''training''}
\begin{usecase}
\addtitle{Training 009}{Actualitzar l'estat d'un ''training''}
\addfield{Actors:}{Sistema}
\addscenario{Curs tipic d'esdeveniments:}{
}
\end{usecase}

\subsection{Llistar prediccions del sistema}
\begin{usecase}
\addtitle{Prediction 001}{Llistar prediccions del sistema}
\addfield{Actors:}{Usuari administrador}
\addscenario{Curs tipic d'esdeveniments:}{
}
\end{usecase}

\subsection{Llistar les meves prediccions}
\begin{usecase}
\addtitle{Prediction 002}{Llistar les meves prediccions}
\addfield{Actors:}{Usuari registrats}
\addscenario{Curs tipic d'esdeveniments:}{
}
\end{usecase}

\subsection{Esborrar una predicci\'{o}}
\begin{usecase}
\addtitle{Prediction 003}{Esborrar una predicci\'{o}}
\addfield{Actors:}{Usuari registrats}
\addscenario{Curs tipic d'esdeveniments:}{
}
\end{usecase}

\subsection{Crear una matriu de predicci\'{o} desde un fitxer}
\begin{usecase}
\addtitle{Prediction 004}{Crear una matriu de predicci\'{o} desde un fitxer}
\addfield{Actors:}{Usuari registrats}
\addscenario{Curs tipic d'esdeveniments:}{
}
\end{usecase}

\subsection{Actualitzar una matriu de predicci\'{o} desde un fitxer}
\begin{usecase}
\addtitle{Prediction 005}{Actualitzar una matriu de predicci\'{o} desde un fitxer}
\addfield{Actors:}{Usuari propietari de la predicci\'{o}}
\addscenario{Curs tipic d'esdeveniments:}{
}
\end{usecase}

\subsection{Configurar un ''sample'' d'una matriu de predicci\'{o}}
\begin{usecase}
\addtitle{Prediction 006}{Configurar un ''sample'' d'una matriu de predicci\'{o}}
\addfield{Actors:}{Usuari propietari de la predicci\'{o}}
\addscenario{Curs tipic d'esdeveniments:}{
}
\end{usecase}

\subsection{Enviar una predicci\'{o} al sistema de cues}
\begin{usecase}
\addtitle{Prediction 007}{Enviar una predicci\'{o} al sistema de cues}
\addfield{Actors:}{Usuari propietari de la predicci\'{o}}
\addscenario{Curs tipic d'esdeveniments:}{
}
\end{usecase}

\subsection{Veure una predicci\'{o}}
\begin{usecase}
\addtitle{Prediction 007}{Enviar una predicci\'{o} al sistema de cues}
\addfield{Actors:}{Usuari propietari de la predicci\'{o}}
\addscenario{Curs tipic d'esdeveniments:}{
}
\end{usecase}

\subsection{Veure els resultats d'una predicci\'{o}}
\begin{usecase}
\addtitle{Prediction 008}{Veure els resultats d'una predicci\'{o}}
\addfield{Actors:}{Usuari propietari de la predicci\'{o}}
\addscenario{Curs tipic d'esdeveniments:}{
}
\end{usecase}

\subsection{Actualitzar l'estat d'una predicci\'{o}}
\begin{usecase}
\addtitle{Prediction 009}{Actualitzar l'estat d'una predicci\'{o}}
\addfield{Actors:}{Sistema}
\addscenario{Curs tipic d'esdeveniments:}{
}
\end{usecase}

\section{Model de dades}
En el seg\�{u}ent gr\`{a}fic veuem  el diagrama UML del modelatge de dades.
\begin{sidewaysfigure}[h]
  \includegraphics[scale=0.5]{img/specification/ModelClass.png}
  \caption{Model de dades}
  \label{specification:datamodel}
\end{sidewaysfigure}