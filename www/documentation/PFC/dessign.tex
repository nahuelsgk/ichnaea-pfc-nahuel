\chapter{Disseny e implementaci\'{o}}
\label{cha:dessign}

En aquest cap\'{i}tol veurem els patrons de dissenys emprats i les tecnologies implementades. Tamb\'{e} descriurem la interf\'{i}cie d'usuari m\'{e}s complexa i el fluxe de les operacions de sistema

\section{Esquema general arquitect\'{o}nic del sistema}
DIAGRAMA DE LA ARQUITECTURA DEL SISTEMA amb els elements WEB APP, SMTP, SISTEMA DE CUES, BASE DE DADES

\section{Patr\'{o} de disseny}
Per la implementaci\`{o} del sistema web s'han usat els seg\�{u}ents patrons de disseny:
  \begin{itemize}
  \item Model-Vista-Controlador amb controlador frontal
  \item Capa de Servei amb injecci\'{o} de depend\`{e}ncies
  \item Mapejat de dades
  \item Repositori de model de dades
  \item View template
  \item Interficies enriquides amb servei webs
  \end{itemize}

\subsection{Esquema del disseny}
DIAGRAMA EXPLICANT EL PATRO DE DISSENY

\section{Implementaci\'{o} i tecnologies}
\subsection{Symfony}
Symfony2 \'{e}s un framework que implementa Model-Vista-Controlador amb controlador frontal amb injecci\'{o} de depencies a la capa de serveis. Les entitats(model de dades) s'han implementat directament amb l'ORM Doctrine contra una base de dades MySQL.
Arquitectonicament, Symfony2 estructura el codi en Bundle. Els "bundles" son un conjunt de serveis, entitats i recursos html independents entre si. Els bundles implementats son
\begin{itemize}
\item{Bundle de usuaris: UserBundle}
\item{Bundle de matrius: MatrixBundle}
\item{Bundle de trainings: TrainingBundle}
\item{Bundle de serveis webs: ApiBundle}
\item{Bundle de predicci\'{o}: PredictionBundle}
\end{itemize}
ESTA LISTA SEGURAMENTE CANVIARA CUANDO SE VAYA REFACTORIZANDO

\subsection{Recursos}
La estructura de recursos \'{e}s la seg\�{u}ent:\\
\begin{itemize}
\item matrix/{id}/
\item matrix/{id}/training/{id}
\item matrix/{id}/training/{id}/prediction/{id}
\end{itemize}

S'ha emprat aquesta estructura de recursos degut a les depencies entre les diferentes entitats. Un training depen d'una matriu i una predicci\'{o} depen d'un ''training''.


\section{Servei web}
S'ha emprat un API JSON Restful per enriquir les interficies. S'ha emprat aquesta tecnologia per la escalabilitat que aporta i perque en un futur es pogui aprofitar el desenvolupament d'aquesta.
Les operacions, els recursos i el parametres son:
\begin{itemize}
\item GET /api/season/{id}   
\item POST /api/season/searchByName
\item GET /api/variable/{variable_id}/seasonSet
\item DELETE /api/variable/{variable_id}/seasonSet/{seasonSet_id}
\item DELETE /api/variable/{variable_id}/seasonSet/{seasonSet_id}/component/{component_id}
\item DELETE /api/variable/{variable_id}/seasonSet/{seasonSet_id}/component/{component_id}/complete
\item PUT /api/matrix/{matrix_id}/column/{column_id}    
\item PUT /api/matrix/{matrix_id}/sample/{sample_id}
\end{itemize}

\section{Capa de serveis}
DESCRIPCI\'{O} DE LA CAPA DE SERVEI. SERVEIS DEFINITS i METODES PRINCIPALS

\section{Integraci\'{o} amb el sistema de cues RabbitMQP}
\subsection{Introducci\'{o} a l'arquitectura de cues}
PETIT DIAGRAMA I EXPLICACI\'{O} DE CONSUMIDORS I PRODUCTORS
\subsection{Consumidors}
\subsubsection{Consumidor ''trainings''}
\subsubsection{Consumidor de prediccions}

\section{Interf\'{i}cies}
\subsection{Interf\'{i}cie de configuraci\'{o} de matrius}




