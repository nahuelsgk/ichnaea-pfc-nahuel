\chapter{Implementaci\'{o}}
\label{cha:implementation}
En aquest capítol es descriuen les tecnologies estudiades i aplicades per implementar el disseny anteriorment descrit.


\section{Estudi previ de les tecnologies}
En aquesta secci\'{o} es far\'{a} una descripci\'{o} de les tecnologies estudiades per tal de poder realitzar aquest projecte. 

\subsection{\textit{Framework} a mida}
En un principi \'{e}s va valorar la possibilitat de desenvolupar tot un nou \textit{framework} per la implementaci\'{o} del projecte. Les tecnologies emprades son PHP com a llenguatge de programaci\'{o} i MySQL com a motor de bases de dades.
\paragraph{Avantatges}
\begin{itemize} 
\item La principal avantatge \'{e}s el control total de la implantaci\'{o} de tots els processos i tecnologies.
\end{itemize} 

\paragraph{Desavantatges}
\begin{itemize} 
\item Desenvolupament extremadament lent.
\item Re-inventar la roda quan no \'{e}s necessari.
\end{itemize} 

\subsection{\textit{Codeigniter}}
Codeigniter \'{e}s un \textit{framework} per a desenvolupament web r\`{a}pid amb PHP.\cite{codeigniter} Usa el patr\'{o} ''active record' per la gesti\'{o} d'emmagazetmatge de dades.\cite{activerecord}
\paragraph{Avantatges}
\begin{itemize}
\item Corba d'aprenentatge rapida.
\item Desenvolupament r\`{a}pid.
\item \'{E}s patr\'{o} MVC.
\end{itemize}

\paragraph{Desavantatges}
\begin{itemize}
\item No compleix tots el requisits de disseny de patrons especificat.
\item No inclou \textit{templating}
\item Dificultat en segregar la l\`{o}gica de la presentaci\'{o}.
\end{itemize}

\subsection{\textit{Symfony2}}
\textit{Symfony2} \'{e}s un HTTP \textit{framework} per a PHP. Nativament implementa una variaci\'{o} del Model-Vista-Controlador amb controlador frontal amb injecci\'{o} de depencies a la capa de serveis.\cite{symfony}

\paragraph{Avantatges}
\begin{itemize}
\item Altament configurable.
\item Compleix tots els requisits de dissenys de patrons especificat.
\item Alt rendiment
\end{itemize}

\paragraph{Desavantatges}
\begin{itemize}
\item Corba d'aprenentatge alta.
\end{itemize}

\section{Implementaci\'{o}}
El \itemit{framework} escollit ha sigut \textit{Symfony2}. A continuaci\'{o} es dona una visi\'{o} general de les tecnologies.

\subsection{\textit{Symfony2}}
Arquitectonicament, \textit{Symfony2} estructura el codi en \textit{Bundles}, similar als paquets de JAVA. Els \textit{bundles} son un conjunt de serveis, entitats i recursos independents entre si. Els \textit{bundles} implementats son:
\begin{itemize}
\item \textit{Bundle} de usuaris: UserBundle. Paquet de serveis, vistes i recursos per usuaris. S'ha utilitzat un paquet de \textit{Symfony2}: FOSUserBundle \cite{fosuserbundle}. S'ha fet una extensi\'{o} del paquet per implementar la gesti\'{o} de rols i grups.
\item \textit{Bundle} de matrius: MatrixBundle. Paquet de serveis, vistes i recursos per les matrius.
\item \textit{Bundle} de trainings: TrainingBundle. Paquet de serveis, vistes i recursos per els entrenaments.
\item \textit{Bundle} de serveis webs: ApiBundle. Paquet de serveis,vistes i recursos per la API JSON Restful. 
\item \textit{Bundle} de predicci\'{o}: PredictionBundle. Paquet de serveis, vistes i recursos per les matrius de prediccions.
\end{itemize}

\subsection{Gesti\'{o} de depend\`{e}ncies} 
\textit{Symfony2} utilitza \textit{Composer} per la gesti\'{o} de depend\`{e}ncies. \textit{Composer} \'{e}s un gestor de dependencies a nivell d'aplicaci\'{o}. Les depend\`{e}ncies son:
\begin{itemize}
\item Jquery: llibreria Javascript que s'executa en el costat client per enriquir les interfícies. Simplifica la manera de interaccionar amb els documents HTML i la simplifica la manera de manipular l'arbre DOM.
\item FOSUserBundle: paquet per a \textit{Symfony} per la gesti\'{o} de usuaris. 
\item Bootstrap: \texit{framework} pel \textit{front-end} desenvolupat per Twitter. Implementa nous estendards com HTML5 i CSS3. Permet el desenvolupament de manera r\`{a}pida i usable interf\'{i}cies soportades per múltiples navegadors.
\item RestBundle: paquet per Symfony per desenvolupar llibreries API JSON Restful.
\item Symfony FS: paquet per gestionar sistema de fitxers.
\item Doctrine Fixtures: paquet per gestionar la inserci\'{o} controlada de dades a la base de dades.
\item TWIG: motor de plantilles per a PHP.
\item Doctrine: ORM pel mapatge de dades, usant MySQL.
\end{itemize}

\subsection{Recursos}
La estructura de recursos de la aplicaci\'{o} \'{e}s la seg\�{u}ent:\\

\begin{tabular}{| l | l |}
\hline
Matrius       & matrix/(id) \\ \hline
''Trainings''  & matrix/(id)/training/(id) \\ \hline
''Predictions'' & matrix/(id)/training/(id)/prediction/(id) \\ \hline
\end{tabular}

S'ha emprat aquesta estructura de recursos degut a les dependències entre les diferents entitats. Un ''training'' dep\'{e}n d'una matriu i una predicci\'{o} dep\'{e}n d'un ''training''.

\section{Servei web}
S'ha desenvolupat una llibreria API JSON Restful\cite{apijson} per enriquir les interfícies. S'ha emprat aquesta tecnologia per la escal.labilitat que aporta i perquè en un futur es pugui aprofitar el desenvolupament d'aquesta.
Les operacions, els recursos i el paràmetres son:\\
\begin{tabular}{ | l | l |}
\hline
GET & /api/season/(id) \\ \hline
POST & /api/season/searchByName \\ \hline
     & INPUT: \{'pattern': string\} \\ \hline
     & OUTPUT: ['string-match-1',...,'string-match-n'] \\ \hline
GET & /api/variable/(id)/seasonSet \\ \hline
DELETE & /api/variable/(id)/seasonSet/(id)\\ \hline
DELETE & /api/variable/(id)/seasonSet/(id) /component/(id) \\ \hline
DELETE & /api/variable/(id)/seasonSet/(id) /component/(id)/complete\\ \hline
PUT & /api/matrix/(id)/column/(id)\\ \hline
PUT & /api/matrix/(id)/sample/(id)\\ \hline
\end{tabular}
Les peticions estan securitzades per ''cookies'' i per sessió per identificar i validar a l'usuari.

\section{Integraci\'{o} amb el sistema de cues RabbitMQP}
\subsection{Introducci\'{o} a l'arquitectura de cues: AMQP}
\label{sec:queue_system_overview}
L'est\`{a}ndard AMQP (''Advanced Message Queuing Protocol'') \'{e}s un protocol d'est\`{a}ndard obert en la capa d'aplicacions d'un sistema de comunicació. Les característiques que definen al protocol AMQP son la orientació a missatges, encuament(''queuing''), enrutament i exactitud, entre altres com,per exemple, la seguretat o les subscripcions.\cite{amqp}\\
AMQP defineix una s\`{e}rie d'entitats. Des de la perspectiva de la interconnexió les m\'{e}s rellevants son:
\begin{itemize}
\item El corredor de missatges: un servidor on els clients AMQP es connecten usant el protocol AMQP. Els corredores de missatges poden executar-se en un entorn distribuït, però aquesta capacitat \'{e}s espec\'{i}fica de la implementació.
\item Usuari: un usuari \'{e}s una entitat amb credencial pot ser autoritzat a connectar-se a un corredor.
\item Connexió: una connexi\'{o} f\'{i}sica, usant per exemple TCP/IP, entre el corredor i l'usuari.
\item Clients: productors i consumidors. EXPLICAR PRODUCTORS I CONSUMIDORS.
\end{itemize}

RabbitMQ \'{e}s un \textit{software} que implementa aquest protocol. El sistema de cues est\'{a} implementat per Miguel Ibero. I ofereix una ofereix una llibreria per la seva integraci\'{o}. 

\subsection{Consumidors i gesti\'{o} de resultats}
Les respostes de les execucions de Ichnaea enviades per la cua (mirar la figura \ref{dessign:archsoftware}), les rep el consumidor. \\
\textit{Symfony2} permet crear comandes CLI per crear processos. Els consumidors necessiten ser processos ''stand-alone''(aut\'{o}noms) que escoltin les sortides, consumeixin, els resultats de la cua. \\
L'esquelet d'una comanda per aquests consumidors actualment \'{e}s:
\begin{lstlisting}
class ConsumerCommand extends ContainerAwareCommand
{
	//Definicio de la comanda
	protected function configure()
	{
		$this
		->setName('nom_de_la_comanda')
		->setDescription('Consumer server');
	}
	
	//Execucio de la comanda
	protected function execute(InputInterface $input, OutputInterface $output)
	{
		//Interficie per integrar el sistema de cues
		$amqp = new AmqpConnection('URI_per_fer_la_connexio');
		$amqp->open();
		
		//Crida al servei
		$servei = $this->getContainer()->get('nom_de_servei');
		
		//Crear el consumidor, amb una 
		//funcio que es crida quan la 
		//cua estableix comunicacio amb 
		//un objecte de la resposta esperada 
		//i li passa el servei 
		$amqp->listenForBuildModelResponse(function (ObjectResponse $resposta_de_la_cua) use ($servei){
			$servei->actualitzaDades($resposta_de_la_cua);
		});
		$amqp->wait();
	}
}
\end{lstlisting}

\subsubsection{Consumidor d'entrenaments}
El consumidor d'entrenaments es queda escoltant les respostes del servei de \\

\subsubsection{Consumidor de prediccions}
El consumidor de prediccions es queda escoltant les respostes del servei de \\

