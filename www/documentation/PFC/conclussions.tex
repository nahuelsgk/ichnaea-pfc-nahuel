\chapter{Conclusions}
\label{cha:conclussions}

En aquest cap\'{i}tol es descriu la evoluci\'{o} temporal, la gesti\{o} del projecte, la estimaci\'{o} econ\'{o}mica i les possibles millores.

\section{Estimaci\'{o} econ\'{o}mica}

\section{Millores en futures versions}
\begin{itemize}
\item Refactoritzaci\'{o} del codi
\item Personalitzaci\'{o} dels perfils d'usuari: formats de dades, dates, idiomes en el cas de internalitzaci\'{o}
\item Ampliaci\'{o} de les API Restful
\item Depuraci\'{o} en els exploradors m\'{e}s habituals. La aplicaci\'{o} est\'{a} depurada per Mozilla Firefox i Google Chrome. Per\'{o} no ha sigut per Internet Explorer o Safari.
\item Depuraci\'{o} en entorn distribu\�{i}t del sistema de cues. Encara que la aplicaci\'{o} ha sigut desenvolupada per ser distribu\�{i}da i separada del sistema de cues, no ha sigut possible probar-la en un entorn m\'{e}s complexe i distribu\�{i}t.
\item Sistema de notificacions. Un sistema m\'{e}s el.laborat de notificacions. Per exemple, notificacions quan un ''training'' ha acabat o enviar
\item Sistema de projectes. La aplicaci\'{o} no ha sigut contemplada per ser m\'{e}s col.laborativa. Hauria de ser m\'{e}s concurrent contra alguns recursos, per exemple diversos editors d'una matriu al mateix temps.
\item Sistema de invitacions. Un sistema per invitar col.laboradors tant siguin de la plataforma com si no.
\item Implementar tests autom\'{a}tics.
\end{itemize}

