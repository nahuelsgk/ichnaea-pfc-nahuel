\chapter{Conclusions}
\label{cha:conclussions}

En aquest cap\'{i}tol es descriu la evolució temporal, la gestió del projecte, la estimació econòmica i les possibles millores.


\section{Metodologia àgil}
Per la realització del projecte s'han emprat un metodologia \texit{agile} i aprofitant tècniques i artefactes de l'SCRUM.\cite{agile} S'han fet iteracions quinzenals amb el ''product owner'' canviant requisits, definint noves especificacions en cadascuna de les reunions, definint nous requisits i fent petites \textit{demos} dels avanços.\\
Els principis bàsics de la metodologia àgil són\cite{agilemanifesto}:
\begin{itemize}
\item Els individus i les seves interaccions per sobre dels processos i les eines.
\item El programari que funciona per sobre de la documentació exhaustiva.
\item La col·laboració amb el client per sobre de la negociació de contractes.
\item La resposta davant del canvi per sobre de seguir un pla tancat.
\end{itemize}

En el cas d'aquest projecte s'ha usat l'artefacte principal ''Backlog'' mitjançant histories de usuaris. Les historia de usuari son una una representació d'un requisit de software escrit en una o dos frases. En aquest cas:\\
\centerline{Com \textbf{rol} vull fer \textbf{alguna cosa} per \textbf{obtenir benefici}}

\subsection{Backlog}
Ens hem basat en les histories d'usuaris amb el següent \textit{backlog} puntuant segons la serie de Fibonnaci.\cite{backlogfibonnacci}

\section{Estimaci\'{o} econ\'{o}mica}


\section{Millores en futures versions}
\begin{itemize}
\item Refactoritzaci\'{o} del codi
\item Personalitzaci\'{o} dels perfils d'usuari: formats de dades, dates, idiomes en el cas de internalitzaci\'{o}
\item Ampliaci\'{o} de les API Restful
\item Depuraci\'{o} en els exploradors m\'{e}s habituals. La aplicaci\'{o} est\'{a} depurada per Mozilla Firefox i Google Chrome. Per\'{o} no ha sigut per Internet Explorer o Safari.
\item Depuraci\'{o} en entorn distribuït del sistema de cues. Encara que la aplicaci\'{o} ha sigut desenvolupada per ser distribuïda i separada del sistema de cues, no ha sigut possible provar-la en un entorn m\'{e}s complexe i distribuït.
\item Sistema de notificacions. Un sistema m\'{e}s el.laborat de notificacions. Per exemple, notificacions quan un ''training'' ha acabat o enviar
\item Sistema de projectes. La aplicaci\'{o} no ha sigut contemplada per ser m\'{e}s col.laborat-iva. Hauria de ser m\'{e}s concurrent contra alguns recursos, per exemple diversos editors d'una matriu al mateix temps.
\item Sistema de invitacions. Un sistema per invitar col.laboradors tant siguin de la plataforma com si no.
\item Implementar tests autom\'{a}tics.
\end{itemize}

