\chapter{Especificaci\'{o}}
\label{cha:background}

\section{Introducci\'{o}}
En aquest cap\´{i}tol es descriu els requeriments i les operatives que es necessiten.
\section{Requeriments funcionals}
\subsection{Administraci\'{o} d'usuaris}
La aplicaci\´{o} ha de estar protegida i autoritzada pels usuaris. Els usuaris autenticats han de tenir permisos i  pertanyer a grups amb roles autoritzats per fer certes accions. Es una aplicaci\´{o} distribuida per tant s'ha de dissenyar un sistema que permeti:
\begin{itemize}
\item Crear comptes
\item Atendre peticions de resetejar contrasenyes
\item Enviar mails de confirmacions de accions
\item Canviar permisos a usuaris
\end{itemize}

\subsection{Administracio de variables}
La aplicaci\´{o} ha de poder crear variables al sistema per poder utilitzar-les al software Ichnaea. Les variables han de tenir asociades de conjunt seasons. Com s'explica al cap\'{i}tol \ref{glossary}, les season es un contingut en format texte que descriu els envelliments de les variables. Per tant s'ha de poder crear afegir conjunts de seasons mitjançant les season sets.

\subsection{Administraci\'{o} de matrius}
La aplicaci\'{o} ha de poder crear i configurar matrius de dades. Aquestes matrius contenen dades que relaciona una variable amb una mostra.
\begin{itemize}
\item Les columnes con les variables de Ichnaea
\end{itemize}

\subsection{Administracio de trainings}

\subsection{Administracio de matrius de prediccio}

\section{Requeriments no funcionals}

\section{Casos d'us i fluxos}

\subsection{Crear un usuari}
\subsection{Canviar un usuari de grup}
\subsection{Crear un matriu desde un csv}
\subsection{Crear una variable}
\subsection{Crear una season set}
\subsection{Clonar una matriu}
\subsection{Configurar un columna de la matriu}
\subsection{Afegir una season a una season set}
\subsection{Configurar un sample}
\subsection{Crear un training}
\subsection{Reenviar un training}
\subsection{Crear una matriu de predicci\'{o}}


\section{Model de dades}