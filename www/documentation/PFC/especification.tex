\chapter{Especificaci\'{o}}
\label{cha:specification}


\section{Introducci\'{o}}
En aquest cap\'{i}tol es descriu els requeriments i les operatives que es necessiten.\\


\section{Requeriments funcionals}

\subsection{Administraci\'{o} d'usuaris}
La aplicaci\'{o} ha de estar protegida i autoritzada pels usuaris. Els usuaris autenticats han de tenir permisos i  pertanyer a grups amb roles autoritzats per fer certes accions. Es una aplicaci\'{o} distribuida per tant s'ha de dissenyar un sistema que permeti:
\begin{itemize}
\item Crear comptes
\item Atendre peticions de resetejar contrasenyes
\item Enviar mails de confirmacions de accions
\item Canviar permisos a usuaris
\end{itemize}

\subsection{Administraci\'{o} de variables}
La aplicaci\'{o} ha de gestionar variables al sistema per poder utilitzar-les al software Ichnaea. Les variables han de tenir asociades de conjunt ''seasons''. Com s'explica al cap\'{i}tol \ref{glossary}, les ''seasons'' son un contingut en format texte que descriu els envelliments de les variables. Per tant s'ha de poder gestionar conjunts de seasons de les season sets.\\

\subsection{Administraci\'{o} de matrius}
La aplicaci\'{o} ha de poder gestionar i configurar matrius de dades. Per la creaci\'{o} de matrius ha de poder llegir un fitxer csv i crear una matriu a partir de les dades proporicionades.
La configuraci\'{o} de matrius ha:
\begin{itemize}
\item Configurar les columnes com una variable i una ''season set''
\item Configurar l'origen de un ''sample''
\item Configurar la data de un ''sample''
\end{itemize}

\subsection{Administraci\'{o} de trainings}
La aplicaci\'{o} ha de:
\begin{itemize}
	\item Gestionar ''trainings'' 
	\item Enviar a processar-los a la cua de processos
	\item Llegir e interpretar el estat del proc\'{e}s 
	\item Llegir els resultats dels ''trainings''
\end{itemize}

\subsection{Administracio de matrius de prediccio}
La aplicaci\'{o} ha de:
\begin{itemize}
	\item Gestionar les matrius de prediccions 
	\item Enviar a processar-los a la cua de processos
	\item Llegir e interpretar el estat del proc\'{e}s 
	\item Llegir els resultats de les prediccions
\end{itemize}

\section{Requeriments no funcionals}
Els requeriments funcionals son:
\begin{itemize}
\item Bon rendiment. La operaci\'{o} m\'{e}s pessada que es la de crear una matriu a partir de un fitxer.
\item Escalabilitat
\item Mantenible
\item Flexibilitat
\end{itemize}


\section{Casos d'us i fluxos}
En la documentaci\'{o} s'utilitzar\`{a} la seg\¨{u}ent estructura:\\
\subsection{Crear un usuari}
\begin{usecase}
\addtitle{Identificador}{Nom cas d'us}
\addfield{Actors:}{Llista de actors}
\addscenario{Curs tipic d'esdeveniments:}{
	\item Esdeveniment
	\item Esdeveniment
 validada.
	\item ...
	
}
\addscenario{Cursos alternatius:}{
\item Esdeveniment Alternatiu
}
\end{usecase}



\subsection{Crear un usuari}
\begin{usecase}
\addtitle{Usuari 001}{Crear un usuari}
\addfield{Actors:}{Anonim}
\addscenario{Curs tipic d'esdeveniments:}{
	\item Usuari accedeix a la plataforma i pot accedir al formulari de registraci\'{o}
	\item Usuari introdueix nom d'usuari, correu electr\`{o}nic i contrasenya.
	\item El sistema envia al usuari una confirmaci\'{o} via correu electr\'{o}nic amb un enllaç de confirmaci\'{o} i crea un compte no validada.
	\item L'usuari rep el correu electr\'{o}nic amb l'enllaç de confirmaci\'{o} i accedeix al l'enllaç de confirmaci\'{o}.
	\item El sistema comprova que \'{e}s un enllaç de confirmaci\'{o} v\`{a}lid i d'aquest usuari i activa la compte. L'usuari ja est\'{a} autenticat al sistema com un usuari i ja \'{e}s un usuari del sistema.
}
\addscenario{Cursos alternatius:}{
\item[3] El sistema valida que no existeixi un usuari amb aquesta compte de correu, que el correu sigui v\`{a}lid. Sino es correcte li informa a l'usuari al mateix formulari.
}
\end{usecase}

\subsection{Canviar un usuari de grup}
\begin{usecase}
\addtitle{Usuari 002}{Canviar un usuari de grup}
\addfield{Actors:}{Anonim}
\addscenario{Curs tipic d'esdeveniments:}{
	\item L'administrador llista tots els usuaris del sistema i selecciona un.
	\item L'administrador veu el formulari de edici\'{o} de permisos.
	\item L'administrador selecciona el nou perm\'{s} i salva el perfil.
	\item El sistema guarda el nou perm\'{i}s
}
\end{usecase}

\subsection{Crear un matriu desde un fitxer}
\begin{usecase}
\addtitle{Matriu 001}{Crear una matrius desde fitxer}
\addfield{Actors:}{Usuari registrat}
\addscenario{Curs tipic d'esdeveniments:}{
	\item L'usuari visualitza el formulari on pot donar nom a la matriu i seleccionar el fitxer en format csv. L'usuari accepta el formulari.
	\item El sistema crear la matriu amb tots els "samples" identificats, les columnes identificades i els origens especificats en el cas que estiguin especificats en el fitxer.
}
\end{usecase}

\subsection{Actualitzar una matriu desde un fitxer}
\begin{usecase}
\addtitle{Matriu 002}{Actualitzar una matriu desde fitxer}
\addfield{Actors:}{Usuari registrat}
\addscenario{Curs tipic d'esdeveniments:}{
	\item L'usuari visualitza el formulari on pot donar nom a la matriu i seleccionar el fitxer en format csv. L'usuari accepta el formulari.
	\item El sistema crear la matriu amb tots els "samples" identificats, les columnes identificades i els origens especificats en el cas que estiguin especificats en el fitxer.
}
\end{usecase}

\subsection{Crear una variable}
\begin{usecase}
\addtitle{Variable 001}{Crear una variable}
\addfield{Actors:}{Usuari administrador}
\addscenario{Curs tipic d'esdeveniments:}{
	\item L'usuari visualitza el formulari on pot donar un identificador que ha de \'{u}nic i una descripci\'{o} .
	\item El sistema crear la variable.
}
\end{usecase}

\subsection{Crear una season set}
\begin{usecase}
\addtitle{Variable 002}{Crear una ''season set''}
\addfield{Actors:}{Usuari registrat}
\addscenario{Curs tipic d'esdeveniments:}{
	\item L'usuari selecciona una variable. 
	\item El sistema renderitza un formulari d'edici\'{o}.
	\item L'usuari accedeix a un formulari de creaci\'{o}
	\item El sistema renderitza un formulari de creaci\'{o}
	\item L'usuari pot donar un nom i seleccionar 0, 1 o 2 fitxers configurats com:
	\begin{itemize}
		\item a unic per tot l'any
		\item com estiu
		\item com hivern
		\item com tardor
		\item com estiu
	\end{itemize}	
}
\end{usecase}


\subsection{Clonar una matriu}
\begin{usecase}
\addtitle{Matriu 002}{Clonar una matriu}
\addfield{Actors:}{Usuari registrat}
\addscenario{Curs tipic d'esdeveniments:}{
	\item L'usuari visualitza un llistat de matrius del sistema i selecciona una matriu per clonar
	\item El sistema renderitza un formulari amb un nom suggerit.
	\item L'usuari pot canviar el nom i acceptar la clonaci\'{o}
	\item El sistema clona la matriu i la seva configuraci\'{o} sense copiar trainigs ni prediccions. El propietari de la matriu \'{e}s l'usuari que ha realitzat la clonaci\'{o}.
}
\end{usecase}

\subsection{Configurar un columna de la matriu}
\begin{usecase}
\addtitle{Matriu 003}{Configurar la columna de una matriu}
\addfield{Actors:}{Usuari propietari de la matriu}
\addscenario{Curs tipic d'esdeveniments:}{
	\item L'usuari selecciona una matrius d'un llistat per configurar-la
	\item El sistema renderitza una vista per configurar les columnes de una matriu.
	\item L'usuari pot canviar d'una columna:
	\begin{itemize}
		\item un alias
		\item seleccionar una variable i una 'season set'
	\end{itemize}
	\item L'usuari accepta la configuraci\'{o}
	\item El sistema salva la configuraci\'{o} de la columna
}
\end{usecase}

\subsection{Afegir una season a una season set}
\begin{usecase}
\addtitle{Matriu 003}{Configurar la columna de una matriu}
\addfield{Actors:}{Administradors?}
\addscenario{Curs tipic d'esdeveniments:}{
	\item L'usuari selecciona una variable i una season set
	\item El sistema renderitza un formulari per gestionar les seasons de la variable i de la season set
	\item L'usuari pot seleccionar un fitxer on cont\'{e} les dades. Pot configurar el fitxer com estiu, primavera, tardor, hivern o unic per tot l'any. L'usuari confirma el fitxer
	\item El sistema guarda el contingut del fitxer i li mostra a l'usuari el llistat actulitzat de seasons
}
\end{usecase}

\subsection{Configurar la data del sample de una matriu}
\begin{usecase}
\addtitle{Sample 001}{Configurar la data del sample de una matriu}
\addfield{Actors:}{Usuari propietari de una matriu}
\addscenario{Curs tipic d'esdeveniments:}{
	\item L'usuari selecciona un matriu i un sample
	\item El sistema mostra un calendari per donar una data.
	\item L'usuari pot seleccionar una data o pot escriure la data. L'usuari confirma la data.
	\item El sistema guarda la data.
}
\end{usecase}

\subsection{Configurar l'origen de una sample}
\begin{usecase}
\addtitle{Sample 001}{Configurar la data del sample de una matriu}
\addfield{Actors:}{Usuari propietari de una matriu}
\addscenario{Curs tipic d'esdeveniments:}{
	\item L'usuari selecciona un matriu i un sample
	\item El sistema mostra un camp per poder escriure.
	\item L'usuari escriu un origen o selecciona un que li suggerireix el sistema. L'usuari confirma la dada.
	\item El sistema guarda la dada.
}
\end{usecase}

\subsection{Llistar trainings}
\begin{usecase}
\addtitle{Training 001}{Llistar trainings}
\addfield{Actors:}{Usuari registrats}
\addscenario{Curs tipic d'esdeveniments:}{
	\item L'usuari accedeix a la vista del llistat de trainings
	\item El sistema llista els trainings amb dades b\`{a}siques. Matriu entrenada, estat del training i descripci\'{o} del training.
}
\end{usecase}

\subsection{Crear un training}
\begin{usecase}
\addtitle{Training 002}{Crear un training}
\addscenario{Curs tipic d'esdeveniments:}{
	\item L'usuari selecciona una matriu per entrenar
	\item El sistema li mostra un formulari per crear trainings
	\item L'usuari pot donar un nom, una descripci\'{o}, seleccionar un origen i quines columnes vol entrenar. Confirma les dades.
	\item El sistema guarda el training i envia al sistema de cues. El sistema evalua si ha pogut enviar el training al sistema de cues en cas que el servei estigui caigut. Renderitza la vista de visualitzaci\'{o} del training.
}
\end{usecase}

\subsection{Reenviar un training}
\begin{usecase}
\addtitle{Training 002}{Reenviar un training}
\addscenario{Curs tipic d'esdeveniments:}{
    \item L'usuari selecciona un training que ha tingut problemes de enviament.
    \item El sistema renderitza una vista de visualitzaci\'{o} del training.
    \item L'usuari pot consultar quin possible error ha passat i pot confirmar el reenviament
    \item El sistema actulitza les dades i reenvia les dades al sistema de cues.
}
\end{usecase}

\subsection{Descarregar els resultats de un training}
\begin{usecase}
\addtitle{Training 003}{Descarregar els resultats de un training}
\addscenario{Curs tipic d'esdeveniments:}{
    \item L'usuari selecciona un training finalitzat
    \item El sistema renderitza una vista de visualitzaci\'{o} del training. Si el resultat es correcte, el sistema mostra una operaci\'{o} de descarrega dels resultats de un training.
    \item L'usuari accedeix a la descarrega.
    \item El sistema envia a l'usuari els resultats.
}
\end{usecase}

\subsection{Visualitzar un training}
\begin{usecase}
\addtitle{Training 004}{Visualitzar un training}
\addscenario{Curs tipic d'esdeveniments:}{
    \item L'usuari selecciona d'un llistat un training.
    \item El sistema renderitza una vista de visualitzaci\'{o} del training amb el nom, descripci\'{o}, data de creaci\'{o} i  errors o resultats segons el cas.
}
\end{usecase}

\subsection{Esborrar un training}
Implementat 50%.

\subsection{Crear una matriu de predicci\'{o}}
\begin{usecase}
\end{usecase}

\subsection{Actualitzar una matriu de predicci\'{o}}

\subsection{Executar una predicci\'{o} de una matriu de predicci\'{o}}

\section{Model de dades}